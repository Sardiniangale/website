\section*{Lesson 3: The Double Pendulum and Generalized Coordinates}

We now consider a more complex system: the double pendulum. It consists of a mass $m_1$ suspended from a fixed pivot by a massless rod of length $l_1$, and a second mass $m_2$ suspended from $m_1$ by a massless rod of length $l_2$. The system moves in a vertical plane.

This system has two degrees of freedom. We can choose the angles $\theta_1$ and $\theta_2$ as our generalized coordinates (coordinate generalizzate). $\theta_1$ is the angle the first rod makes with the vertical, and $\theta_2$ is the angle the second rod makes with the vertical.

The Cartesian positions of the two masses can be expressed in terms of these generalized coordinates. Let the pivot be at the origin $(0,0)$.
The position of the first mass, $m_1$, is:
\begin{align*}
    x_1 &= l_1\sin\theta_1 \\
    y_1 &= -l_1\cos\theta_1
\end{align*}
The position of the second mass, $m_2$, is found by adding the displacement from $m_1$:
\begin{align*}
    x_2 &= x_1 + l_2\sin\theta_2 = l_1\sin\theta_1 + l_2\sin\theta_2 \\
    y_2 &= y_1 - l_2\cos\theta_2 = -l_1\cos\theta_1 - l_2\cos\theta_2
\end{align*}
The entire state of the system at any time is completely determined if we know the values of $\theta_1(t)$ and $\theta_2(t)$.

The goal of Lagrangian mechanics is to describe the dynamics of the system using these generalized coordinates. The method involves two key scalar functions: the kinetic energy (energia cinetica), $T$, and the potential energy (energia potenziale), $U$.

The kinetic energy is the sum of the kinetic energies of the two masses, $T = T_1 + T_2 = \frac{1}{2}m_1v_1^2 + \frac{1}{2}m_2v_2^2$. To calculate this, we need the velocities, which are the time derivatives of the positions. The velocities will depend on $\theta_1, \theta_2, \dot{\theta}_1, \dot{\theta}_2$. Therefore, the total kinetic energy is a function of the generalized coordinates and their time derivatives:
\[
    T = T(\theta_1, \theta_2, \dot{\theta}_1, \dot{\theta}_2)
\]
The potential energy depends on the vertical positions of the masses, $U = U_1 + U_2 = m_1gy_1 + m_2gy_2$. It is a function of the generalized coordinates only:
\[
    U = U(\theta_1, \theta_2)
\]
In classical mechanics, the dynamics of many systems can be elegantly described by a single function called the Lagrangian (Lagrangiana), $L$, which is defined as the difference between the kinetic and potential energy:
\[
    L = T - U
\]
For the double pendulum, the Lagrangian is $L(\theta_1, \theta_2, \dot{\theta}_1, \dot{\theta}_2)$.

The equations of motion for the system are obtained from the Lagrangian using the principle of stationary action, which leads to the Euler-Lagrange equations. There will be one equation for each generalized coordinate. The derivation and application of these equations will be the subject of the next lesson.
