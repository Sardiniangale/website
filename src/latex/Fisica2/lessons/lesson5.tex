\section*{Lesson 5: Maxwell's First Equation and Electrostatic Potential}

This lesson begins by deriving the first of Maxwell's Equations, which concerns the divergence of the electric field. Starting from the divergence theorem (teorema della divergenza), which relates the flux of a vector field through a closed surface to the integral of its divergence over the enclosed volume:
\begin{equation}
    \oint_S \vec{E} \cdot d\vec{S} = \int_\tau (\nabla \cdot \vec{E}) \, d\tau
\end{equation}
We can combine this with Gauss's Law (Teorema di Gauss), which states that the flux of the electric field through a closed surface is proportional to the enclosed charge \(Q_{\text{int}}\):
\begin{equation}
    \oint_S \vec{E} \cdot d\vec{S} = \frac{Q_{\text{int}}}{\epsilon_0}
\end{equation}
By equating these two expressions for the flux, and expressing the enclosed charge as the integral of the volume charge density (densità volumetrica di carica) \(\rho\), we get:
\begin{equation}
    \int_\tau (\nabla \cdot \vec{E}) \, d\tau = \int_\tau \frac{\rho}{\epsilon_0} \, d\tau
\end{equation}
Since this equality must hold for any arbitrary volume \(\tau\), the integrands themselves must be equal. This gives us the differential form of Gauss's Law, also known as the first of Maxwell's Equations for the electric field in a vacuum:
\begin{equation}
    \nabla \cdot \vec{E} = \frac{\rho}{\epsilon_0}
\end{equation}
This is a local equation (equazione locale) because it relates the divergence of the electric field at a specific point in space to the charge density at that very same point. This is in contrast to the integral form of Gauss's Law, which is non-local as it relates the field on a surface to the total charge contained within it.

To handle point charges (cariche puntiformi) within the framework of continuous charge densities, we introduce the Dirac delta function (delta di Dirac), which is technically a distribution (distribuzione). It is defined as being zero everywhere except at the origin, where it is infinite, yet its integral over all space is one. This allows us to define the volume charge density for a point charge \(q_0\) located at position \(\vec{r}_0\) as:
\begin{equation}
    \rho(\vec{r}) = q_0 \delta(\vec{r} - \vec{r}_0)
\end{equation}

Next, we explore the line integral (integrale di linea) of the electrostatic field between two points, A and B. For the field generated by a single point charge, the result of this integral depends only on the start and end points, not on the path taken. A field with this property is called a conservative field (campo conservativo). A direct consequence is that the line integral of the electrostatic field around any closed loop is zero:
\begin{equation}
    \oint \vec{E} \cdot d\vec{l} = 0
\end{equation}
This conservative nature allows us to define a scalar field called the electrostatic potential (potenziale elettrostatico), V. The potential difference (differenza di potenziale) between two points is related to the line integral of the electric field:
\begin{equation}
    V(A) - V(B) = \int_A^B \vec{E} \cdot d\vec{l}
\end{equation}
From this relationship, we can derive the differential form, which connects the electric field to the potential via the gradient (gradiente):
\begin{equation}
    \vec{E} = -\nabla V
\end{equation}
The electric field points from regions of higher potential to regions of lower potential.

Surfaces on which the potential V is constant are known as equipotential surfaces (superfici equipotenziali). The electric field lines are always perpendicular to these surfaces. No work (lavoro) is done when moving a charge along an equipotential surface, as the displacement is always orthogonal to the electric force.

Finally, we define the electrostatic potential energy (energia potenziale elettrostatica) U of a point charge q placed in a potential V as \(U = qV\). This energy can be interpreted as the work done by an external force to bring the charge from a reference point (where the potential is defined as zero) to its current position. The unit of potential is the Volt (Volt), defined as a Joule per Coulomb. In atomic and nuclear physics, a convenient unit of energy is the electron-volt (elettronvolt), which is the energy gained by an electron when it moves through a potential difference of one volt.