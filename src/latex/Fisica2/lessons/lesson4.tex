\section*{Lesson 4: Superposition, Conductors, and Plasma Oscillations}

The superposition principle (principio di sovrapposizione) is a fundamental concept for calculating electric fields from complex charge distributions. A powerful illustration is the case of two overlapping spheres of the same radius $R$, one with uniform positive charge density $+\rho_0$ and the other with uniform negative charge density $-\rho_0$. The total electric field is the vector sum of the fields from each sphere. Using the known result for the field inside a uniformly charged sphere, $\vec{E} = \frac{\rho_0}{3\epsilon_0}\vec{r}$, we can calculate the field in the overlapping region. If the centers of the spheres are separated by a vector $\vec{\delta}$, the field in the overlapping region is found to be uniform and constant:
\begin{equation}
\vec{E}_{overlap} = -\frac{\rho_0}{3\epsilon_0}\vec{\delta}
\end{equation}

This model provides an excellent analogy for the behavior of conductors (conduttori) in an external electric field. In a conductor, conduction electrons are free to move. When an external field $\vec{E}_{ext}$ is applied, the "sea" of electrons displaces slightly, creating a charge separation, or polarization (polarizzazione). This separation induces an internal electric field (campo elettrico interno) $\vec{E}_{int}$ that opposes the external field. The displacement continues until the net electric field inside the conductor is zero, a condition known as electrostatic equilibrium (equilibrio elettrostatico). At this point, $\vec{E}_{ext} + \vec{E}_{int} = 0$. The conductor effectively shields its interior from the external static electric field.

If the external field is suddenly removed, the displaced electron sea is no longer in equilibrium and experiences a restoring force from the fixed positive ions. This force acts like a spring, leading to oscillations. The equation of motion for the electron sea is that of a simple harmonic oscillator (oscillatore armonico semplice). This collective, high-frequency oscillation of the electrons is a quantum mechanical effect known as ...IDK this ???. The characteristic angular frequency of this oscillation is the plasma frequency, $\omega_p$, a property of the material that depends on the density of conduction electrons.
\begin{equation}
\omega_p^2 = \frac{n_e e^2}{m_e\epsilon_0}
\end{equation}
This phenomenon is crucial in understanding the optical properties of metals.
