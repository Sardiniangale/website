\section{The Electric Field}

The electric field (\textit{campo elettrico}) is a vector field representing the force per unit charge. It extends the concept of Coulomb's force (\textit{forza di Coulomb}) to describe the influence of charges in space.

The electric field \(\vec{E}\) at a point is defined as the force \(\vec{F}\) on a test charge (\textit{carica di prova}) \(q_0\) at that point, divided by the charge. The limit as \(q_0 \to 0\) is taken to not disturb the source charge:
\begin{equation}
    \vec{E} = \lim_{q_0 \to 0} \frac{\vec{F}}{q_0}
\end{equation}
The unit is Newtons per Coulomb (N/C), or Volts per meter (V/m).

For a single point charge (\textit{carica puntiforme}) \(Q\) at the origin, the electric field is:
\begin{equation}
    \vec{E}(\vec{r}) = \frac{1}{4\pi\epsilon_0} \frac{Q}{r^2} \hat{r}
\end{equation}
where \(\hat{r}\) is the radial unit vector (\textit{versore}). For a charge \(Q_1\) at \(\vec{r}_1\), the field at \(\vec{r}\) is:
\begin{equation}
    \vec{E}(\vec{r}) = \frac{1}{4\pi\epsilon_0} \frac{Q_1}{|\vec{r} - \vec{r}_1|^3} (\vec{r} - \vec{r}_1)
\end{equation}

The superposition principle (\textit{principio di sovrapposizione}) states that the total electric field from multiple charges is their vector sum.

Field lines (\textit{linee di campo}) are used to visualize the electric field. They are tangent to the field vector at every point, originate from positive charges, and terminate on negative charges or at infinity. Their density indicates the field's strength.

For continuous charge distributions (\textit{distribuzioni continue di carica}) (linear \(\lambda\), surface \(\sigma\), or volume \(\rho\)), the field is the integral of the contributions from each infinitesimal charge element. For a volume distribution:
\begin{equation}
    \vec{E}(\vec{r}) = \frac{1}{4\pi\epsilon_0} \int_V \frac{\rho(\vec{r}')}{|\vec{r} - \vec{r}'|^3} (\vec{r} - \vec{r}') dV'
\end{equation}

For an infinite straight wire (\textit{filo rettilineo infinito}) with uniform charge density \(\lambda\), the field is \(E = \frac{\lambda}{2\pi\epsilon_0 z}\) at distance \(z\). For a charged ring (\textit{anello carico}) of radius \(R\) and charge \(Q\), the axial field is \(E_z = \frac{1}{4\pi\epsilon_0} \frac{Qz}{(z^2 + R^2)^{3/2}}\). For a uniformly charged disk (\textit{disco carico}) of radius \(R\) and density \(\sigma\), the axial field is \(E_z = \frac{\sigma}{2\epsilon_0} \left(1 - \frac{z}{\sqrt{z^2 + R^2}}\right)\). For an infinite plane (\textit{piano infinito}), the field is uniform (\textit{campo uniforme}): \(E = \frac{\sigma}{2\epsilon_0}\).

Gauss's Law (\textit{teorema di Gaus}) relates the electric flux (\textit{flusso}), \(\Phi_E\), through a closed surface (\textit{superficie chiusa}) to the enclosed charge \(Q_{int}\):
\begin{equation}
    \Phi_E = \oint_S \vec{E} \cdot d\vec{S} = \frac{Q_{int}}{\epsilon_0}
\end{equation}
where \(Q_{int}\) is the total internal charge (\textit{cariche interne}).
