\section*{Lesson 1: Coulomb's Law}

Coulomb's Law (Legge di Coulomb) describes the electrostatic force (forza elettrostatica) between two static charged particles (particelle cariche). The force \(\vec{F}_{12}\) that a charge (carica) \(q_{1}\) exerts on a second charge \(q_{2}\) is directly proportional to the product of the charges and inversely proportional to the square of the distance between them. The force is directed along the line connecting the two charges.

The vector form of the equation is:
\begin{equation} 
    \vec{F}_{12} = k \frac{q_1 q_2}{R^2} \hat{\vec{R}}_{12}
\end{equation}
Here, \(k\) is Coulomb's constant (costante di Coulomb), \(R\) is the distance between the charges, and \(\hat{\vec{R}}_{12}\) is the unit vector (versore) pointing from \(q_{1}\) to \(q_{2}\). The law is reciprocal, meaning the force exerted by \(q_{2}\) on \(q_{1}\) is equal in magnitude and opposite in direction, so \(\vec{F}_{12} = -\vec{F}_{21}\).

Coulomb's constant is expressed in terms of the vacuum permittivity (permittività del vuoto), \(\epsilon_{0}\), as:
\begin{equation} 
    k = \frac{1}{4\pi\epsilon_{0}}
\end{equation}
Following the 2019 redefinition of the SI base units (Unità di base del SI), the elementary charge (carica elementare) \(e\) is a defining constant. As a result, \(\epsilon_{0}\) is now an experimentally determined value with an associated uncertainty, not a defined quantity.

The electric force is significantly stronger than the gravitational force. For instance, the electric force (forza elettrica) \(F_{E}\) between a proton and an electron is many orders of magnitude greater than the gravitational force (forza gravitazionale) \(F_{G}\) between them:
\begin{align}
    F_G &= G \frac{m_p m_e}{R^2} \approx 10^{-47} N \\
    F_E &= \frac{1}{4\pi\epsilon_{0}} \frac{e^2}{R^2} \approx 10^{-7} N
\end{align}
