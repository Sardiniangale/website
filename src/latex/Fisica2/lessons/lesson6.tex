\section*{Lesson 6: Electrostatic Potential Energy and The Electric Dipole}

\subsection*{Electrostatic Potential Energy}

We continue our discussion on electrostatic potential energy. For a single point charge \(Q\), the electrostatic energy is given by the product of the charge and the potential \(V\) at its location: \(U = Q \cdot V\). This corresponds to the negative of the work done by the field's forces to bring the charge from infinity to that point, denoted as \(L_{\infty}\).

Let's extend this to a system of two point charges, \(Q_1\) and \(Q_2\), subject to the Coulomb force. To assemble this configuration, we can imagine bringing the charges from infinity. Placing the first charge, \(Q_1\), requires no work as there is no pre-existing electric field. However, positioning the second charge, \(Q_2\), involves work done by the electric field generated by \(Q_1\).

The work \(L_{\text{field}}\) done by the field to bring \(Q_2\) from infinity to a distance \(R_{12}\) from \(Q_1\) is calculated as:
\[ L_{\text{field}} = \int_{\infty}^{R_{12}} \vec{F} \cdot d\vec{l} = \int_{\infty}^{R_{12}} \frac{Q_1 Q_2}{4\pi\epsilon_0 R^2} \hat{R} \cdot d\vec{l} \]
Since the electrostatic field is conservative, the integral is path-independent. We can choose a radial path, so \(d\vec{l} = dR \hat{R}\).
\[ L_{\text{field}} = \int_{\infty}^{R_{12}} \frac{Q_1 Q_2}{4\pi\epsilon_0 R^2} dR = \left[ -\frac{Q_1 Q_2}{4\pi\epsilon_0 R} \right]_{\infty}^{R_{12}} = -\frac{Q_1 Q_2}{4\pi\epsilon_0 R_{12}} \]
The external work \(L_{\text{ext}}\) required to assemble the configuration at a constant, near-zero velocity (implying zero change in kinetic energy, \(\Delta K = 0\)) is the negative of the work done by the field: \(L_{\text{ext}} = -L_{\text{field}}\).
\[ L_{\text{ext}} = \frac{Q_1 Q_2}{4\pi\epsilon_0 R_{12}} \]
The potential energy \(U_2\) of the two-charge system is defined as this external work, which is equivalent to the negative of the work done by the field:
\[ U_2 = -L_{\text{field}} = \frac{Q_1 Q_2}{4\pi\epsilon_0 R_{12}} \]

For a system of three point charges (\(Q_1, Q_2, Q_3\)), the total potential energy is the sum of the work required to assemble them. We first bring in \(Q_1\) (zero work), then \(Q_2\) (work \(U_2\)), and finally \(Q_3\) in the presence of the fields from \(Q_1\) and \(Q_2\). The external work to bring in \(Q_3\) is:
\[ L_3 = \frac{Q_1 Q_3}{4\pi\epsilon_0 R_{13}} + \frac{Q_2 Q_3}{4\pi\epsilon_0 R_{23}} \]
The total potential energy \(U_3\) of the system is the sum of the energies of all pairs:
\[ U_3 = U_2 + L_3 = \frac{1}{4\pi\epsilon_0} \left( \frac{Q_1 Q_2}{R_{12}} + \frac{Q_1 Q_3}{R_{13}} + \frac{Q_2 Q_3}{R_{23}} \right) \]

This can be generalized for a system of N point charges. The total electrostatic potential energy is the sum over all unique pairs of charges:
\[ U_N = \frac{1}{2} \sum_{i=1}^{N} \sum_{j=1, j \neq i}^{N} \frac{Q_i Q_j}{4\pi\epsilon_0 R_{ij}} \]
The factor of \(\frac{1}{2}\) is included to correct for double-counting each pair in the summation (e.g., the pair \(Q_1, Q_2\) is counted as both \(i=1, j=2\) and \(i=2, j=1\)).
This expression can be rewritten as:
\[ U_N = \frac{1}{2} \sum_{i=1}^{N} Q_i \left( \sum_{j=1, j \neq i}^{N} \frac{Q_j}{4\pi\epsilon_0 R_{ij}} \right) \]
The term in the parenthesis is the electric potential \(V_i\) at the position of charge \(Q_i\) due to all other charges. Therefore:
\[ U_N = \frac{1}{2} \sum_{i=1}^{N} Q_i V_i \]
This represents the interaction energy (energia di interazione) of the N point charges. It is the energy required to assemble the configuration of charges, assuming they are pre-existing.

For continuous charge distributions, the sum becomes an integral. The potential energy \(U\) for a volume charge distribution with density \(\rho\) is:
\[ U = \frac{1}{2} \int_V \rho(\vec{r}) V(\vec{r}) d\tau \]
where the integral is over the volume \(V\) where the charge density \(\rho\) is non-zero.
For surface and linear distributions, the expressions are analogous:
\[ U = \frac{1}{2} \int_S \sigma V dS \quad \text{and} \quad U = \frac{1}{2} \int_L \lambda V dl \]
This formulation for continuous distributions represents the total electrostatic potential energy, which includes both the interaction energy between different infinitesimal charge elements and the self-energy (autoenergia) required to assemble each infinitesimal charge element itself.

As an example, the energy of a uniformly charged sphere of radius \(R\) and total charge \(Q\) can be calculated. The result is:
\[ U = \frac{1}{4\pi\epsilon_0} \frac{3}{5} \frac{Q^2}{R} \]
This result can be used to define a conventional radius for the electron, known as the classical electron radius (raggio classico dell'elettrone). By equating the electrostatic energy of a sphere with charge \(e\) to the electron's rest energy \(m_e c^2\), we can solve for the radius. Omitting the \(\frac{3}{5}\) factor for an order-of-magnitude estimate:
\[ m_e c^2 \approx \frac{1}{4\pi\epsilon_0} \frac{e^2}{r_e} \implies r_e = \frac{1}{4\pi\epsilon_0} \frac{e^2}{m_e c^2} \approx 2.8 \times 10^{-15} \, \text{m} \]

We can express the electrostatic energy in terms of the electric field. Starting from \(U = \frac{1}{2} \int \rho V d\tau\) and using Maxwell's first equation, \(\nabla \cdot \vec{E} = \rho / \epsilon_0\), we substitute \(\rho = \epsilon_0 \nabla \cdot \vec{E}\):
\[ U = \frac{\epsilon_0}{2} \int_V (\nabla \cdot \vec{E}) V d\tau \]
Using the vector identity \(\nabla \cdot (V\vec{E}) = (\nabla V) \cdot \vec{E} + V(\nabla \cdot \vec{E})\) and the relation \(\vec{E} = -\nabla V\), we get \(V(\nabla \cdot \vec{E}) = \nabla \cdot (V\vec{E}) + E^2\). Substituting this into the energy integral:
\[ U = \frac{\epsilon_0}{2} \int_V (\nabla \cdot (V\vec{E}) + E^2) d\tau = \frac{\epsilon_0}{2} \int_V \nabla \cdot (V\vec{E}) d\tau + \frac{\epsilon_0}{2} \int_V E^2 d\tau \]
Applying the divergence theorem to the first term converts the volume integral into a surface integral over the boundary \(S\) of the volume \(V\):
\[ \int_V \nabla \cdot (V\vec{E}) d\tau = \oint_S (V\vec{E}) \cdot d\vec{S} \]
If we extend the integration volume to all of space, the surface \(S\) goes to infinity. At large distances \(r\) from a finite charge distribution, \(V\) falls off as \(1/r\) and \(E\) as \(1/r^2\), while the surface area \(S\) grows as \(r^2\). The product \(V\vec{E}\) thus decreases as \(1/r^3\), and the surface integral \(\oint (V\vec{E}) \cdot d\vec{S} \propto (1/r^3) \cdot r^2 = 1/r\), which tends to zero as \(r \to \infty\).
Therefore, the first term vanishes, and the total energy is:
\[ U = \frac{\epsilon_0}{2} \int_{\text{all space}} E^2 d\tau \]
This allows us to define the electrostatic energy density (densità di energia elettrostatica) as:
\[ u_e = \frac{1}{2} \epsilon_0 E^2 \]
This interpretation is conceptually powerful, as it associates energy with the electric field itself, distributed throughout space. The total energy is the integral of this density over all space where the field exists.

\subsection*{The Electric Dipole}

An electric dipole (dipolo elettrico) is a system of two equal and opposite point charges, \(+Q\) and \(-Q\), separated by a small distance vector \(\vec{\delta}\), which points from the negative to the positive charge by convention.
The electric dipole moment (momento di dipolo elettrico) is defined as:
\[ \vec{p} = Q \vec{\delta} \]
The unit is Coulomb-meter (C·m). Dipoles are fundamental in physics and chemistry. Many molecules, like water (H\(_2\)O), are polar molecules, meaning they have a permanent electric dipole moment. This explains phenomena like water's high boiling point and its effectiveness as a solvent. The forces between dipoles, such as hydrogen bonds (legami idrogeno), are crucial for the structure of DNA and proteins.

The electric potential \(V\) at a point P far from a dipole (\(r \gg \delta\)) can be approximated. The exact potential is the sum of the potentials from the two charges:
\[ V(\vec{r}) = \frac{Q}{4\pi\epsilon_0 r_+} - \frac{Q}{4\pi\epsilon_0 r_-} = \frac{Q}{4\pi\epsilon_0} \frac{r_- - r_+}{r_- r_+} \]
For large distances \(r\), we can approximate \(r_- r_+ \approx r^2\) and \(r_- - r_+ \approx \delta \cos\alpha\), where \(\alpha\) is the angle between the dipole moment \(\vec{p}\) and the position vector \(\vec{r}\).
\[ V(\vec{r}) \approx \frac{Q \delta \cos\alpha}{4\pi\epsilon_0 r^2} = \frac{p \cos\alpha}{4\pi\epsilon_0 r^2} \]
This can be written compactly using the dot product:
\[ V(\vec{r}) = \frac{\vec{p} \cdot \vec{r}}{4\pi\epsilon_0 r^3} \]
The potential of a dipole falls off as \(1/r^2\). The electric field \(\vec{E}\) is found by taking the negative gradient of the potential, \(\vec{E} = -\nabla V\). In spherical coordinates, with \(\vec{p}\) along the z-axis, the potential is \(V(r, \theta) = \frac{p \cos\theta}{4\pi\epsilon_0 r^2}\). The field components are:
\[ E_r = -\frac{\partial V}{\partial r} = \frac{2p \cos\theta}{4\pi\epsilon_0 r^3} \]
\[ E_\theta = -\frac{1}{r}\frac{\partial V}{\partial \theta} = \frac{p \sin\theta}{4\pi\epsilon_0 r^3} \]
\[ E_\phi = 0 \]
The electric field of a dipole falls off as \(1/r^3\), which is faster than the \(1/r^2\) decay of a single point charge. This is because, from far away, the two opposite charges appear to cancel each other out.

A dipole placed in a uniform external electric field \(\vec{E}_{\text{ext}}\) experiences no net force, as the forces on the positive and negative charges are equal and opposite (\(\vec{F}_{\text{net}} = Q\vec{E}_{\text{ext}} - Q\vec{E}_{\text{ext}} = 0\)). However, it does experience a torque (momento delle forze):
\[ \vec{\tau} = \vec{p} \times \vec{E}_{\text{ext}} \]
This torque tends to align the dipole moment with the external field.

In a non-uniform electric field, the forces on the two charges are no longer equal, resulting in a net force on the dipole. This force can be expressed as:
\[ \vec{F} = (\vec{p} \cdot \nabla) \vec{E}_{\text{ext}} \]
This is also equal to the gradient of the scalar product of \(\vec{p}\) and \(\vec{E}\):
\[ \vec{F} = \nabla(\vec{p} \cdot \vec{E}_{\text{ext}}) \]

The potential energy \(U\) of a dipole in an external electric field is the work required to orient it in the field. It is given by:
\[ U = - \vec{p} \cdot \vec{E}_{\text{ext}} \]
The energy is minimized (\(U = -pE\)) when the dipole is aligned with the field (\(\theta = 0\)), which is a stable equilibrium. The energy is maximized (\(U = +pE\)) when it is anti-aligned (\(\theta = \pi\)), which is an unstable equilibrium. The force on the dipole can also be derived from the potential energy: \(\vec{F} = -\nabla U = \nabla(\vec{p} \cdot \vec{E}_{\text{ext}})\).
