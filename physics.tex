%% LyX 2.4.3 created this file.  For more info, see https://www.lyx.org/.
%% Do not edit unless you really know what you are doing.
\documentclass[american]{article}
\usepackage[T1]{fontenc}
\usepackage[utf8]{luainputenc}
\usepackage{cancel}
\usepackage{esint}
\usepackage{babel}
\begin{document}
\title{Physics}
\date{March 25, 2025}
\author{Giacomo}
\maketitle

\part*{Early Kinematics}

\part*{SDR and Galilean relativity }

\part*{Late Kinematics}

\part*{Special Relativity}

\part*{Thermodynamics}

\textquotedbl It’s no use going back to yesterday, because I was
a different person then.\textquotedbl{}

--- Alice, Alice in Wonderland

\subsection*{The zeroth law of thermodynamics}

\subsubsection*{Macrostates and Microstates}

A Macrostate is a macroscopic description of a system using the 4
main thermodynamic variables pressure (p), temperature (T), volume
(V), and number of particles (N). These variables are averages e.g
the pressure of a gas is the average force per unit area exerted by
particles, temperature reflects the average kinetic energy of particles.
A Microstate is a complete, detailed description of every particle
in the system, including their positions and momenta. MICROSTATES
WILL NOT BE COVERED.

\subsubsection*{The zeroth law}

\paragraph*{``If A, B and C are different thermodynamical systems and A is in
thermodynamical equilibrium with B, and B is in thermodynamical equilibrium
with, then A is in the themodynamical equilibrium with C''}

\subsubsection*{The first law}

\paragraph*{``The internal energy of an isolated system is conserved under any
thermodynamical change''}

\subsubsection*{The second law}

\paragraph*{Under any thermodynamical change:}

\[
\Delta U=Q+W
\]
 And in differential form using inexact derivatives

\[
dU=\mkern3mu\mathchar'26\mkern-12mu dQ+\mkern3mu\mathchar'26\mkern-12mu dW
\]

-This will be explained further on.

\subsubsection*{Work}

Work is energy being transferred between one system. Due to this it
cannot be tied to a thermodynamical process, and therefore is a function
of path. A system doesn't store work, therefore it cant be measured
in the exact differential equation at the start or end of a reaction.
This is why it is represented in a inexact differential written as
đX, which allows us to show work transforming over time. 

Work can be defined as:

\begin{equation}
dW=\vec{F}\times\vec{d}h
\end{equation}
 For the first law of thermodynamics, W > 0 if work is done on the
system, such as compression and W < 0 work done by the system, usually
expansion.

\paragraph{Pressure and Work. }

If work is done by the system us negative F, if instead its work done
on the system the F is positive.

\[
dW=\pm Fdh
\]
Force in terms of pressure is determined by:

\[
F=p_{surr}\times A
\]
 Where A is the relevant surface area

To represent the change in volume 

\[
dV=A\times dh\Longrightarrow dh=\frac{dV}{A}
\]
 Now when we substitute F and dh

\[
dW=\pm(p_{surr}\times A)\left(\frac{dV}{A}\right)
\]

\[
dW=-p_{surr}dV
\]
 Therefore the total work is the integral of dW:

\[
W=\int_{V_{i}}^{V_{f}}\pm p_{surr}dV
\]


\subsubsection*{Heat}

Heat is calculated in a much easier manner. Using the second law of
thermodynamics:

\[
\Delta U=Q+W
\]
 Solve for Q

\[
Q=\Delta U-W
\]
 Keep in mind that this is only valid for closed systems.

\paragraph{Heat capacity at a constant volume ($C_{V}$).}

For context the internal energy for ideal gas is:

\[
U=\frac{3}{2}nRT
\]
 I will explain this better later, but for reference n is number of
moles and R is the gas constant.

Lets take the first law, where đW is 0 because the volume is constant.

\[
dU=\mkern3mu\mathchar'26\mkern-12mu dQ+\cancelto{0}{\mkern3mu\mathchar'26\mkern-12mu dW}
\]

\[
C_{V}=\left(\frac{\partial Q}{\partial T}\right)_{V}=\left(\frac{\partial U}{\partial T}\right)_{V}
\]

\[
C_{V}=\frac{\partial}{\partial T}\left(\frac{3}{2}nRT\right)_{V}
\]
 
\[
C_{V}=\frac{3}{2}nR
\]


\paragraph*{Heat capacity at a constant pressure ($C_{P}$).}

For this problem work is done because gas can expand. Since $\mkern3mu\mathchar'26\mkern-12mu dW=pdV$

\[
dU=\mkern3mu\mathchar'26\mkern-12mu dQ+\cancelto{pdV}{\mkern3mu\mathchar'26\mkern-12mu dW}
\]

\[
C_{p}=\left(\frac{\partial Q}{\partial T}\right)_{p}=\left(\frac{\partial U}{\partial T}\right)_{p}+p\left(\frac{\partial V}{\partial T}\right)_{p}
\]
 Because we are working with a ideal gas, U is dependent on only T
so we can simplify it $C_{V}$.

THIS SECTION NEEDS WORK, MISSING A LOT OF CRITICAL STEPS

\[
V=\frac{nRT}{p}
\]

\[
\left(\frac{\partial V}{\partial T}\right)_{p}=\frac{nR}{p}
\]

\[
C_{P}=C_{V}+p\left(\frac{nR}{p}\right)
\]

\[
C_{p}=C_{V}+nR
\]

\[
C_{p}=\frac{3}{2}nR+nR
\]

\[
C_{p}=\frac{5}{2}nR
\]


\subsubsection*{Internal Energy}

\subsubsection*{Thermodynamic processes}
\end{document}
