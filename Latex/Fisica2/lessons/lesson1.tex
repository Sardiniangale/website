\section*{Class Information}

\subsection*{Recommended Textbooks}
\begin{itemize}
    \item The Feynman Lectures on Physics
    \item Picasso - Lezioni di fisica
\end{itemize}

\section{Coulomb's Law}

Coulomb's law describes the electrostatic force between two charged particles. The force $\mathbf{F}_{12}$ exerted by a charge $q_1$ on a charge $q_2$ is given by:
\begin{equation}
    \mathbf{F}_{12} = k \frac{q_1 q_2}{R^2} \hat{\mathbf{R}}_{12}
\end{equation}
where $k$ is Coulomb's constant, $R$ is the distance between the charges, and $\hat{\mathbf{R}}_{12}$ is the unit vector pointing from $q_1$ to $q_2$. The force exerted by $q_2$ on $q_1$ is equal and opposite:
\begin{equation}
    \mathbf{F}_{12} = -\mathbf{F}_{21}
\end{equation}

Coulomb's constant $k$ is defined as:
\begin{equation}
    k = \frac{1}{4\pi\epsilon_0}
\end{equation}
where $\epsilon_0$ is the permittivity of free space.

\section{2019 Redefinition of SI Base Units}

Since 2019, the SI base units are defined by setting the numerical values of seven defining constants. This includes the elementary charge, $e$. As a result, the vacuum permittivity $\epsilon_0$ is no longer a defined constant but is a measured value with an associated uncertainty. This is a change from the pre-2019 definition where $\epsilon_0$ was an exact value.

The redefinition emphasizes the fundamental constants of nature. For example, the gravitational force $F_G$ and the electric force $F_E$ can be compared:
\begin{align}
    F_G &= G \frac{m_p m_e}{R^2} \approx 10^{-47} N \\
    F_E &= \frac{1}{4\pi\epsilon_0} \frac{e^2}{R^2} \approx 10^{-7} N
\end{align}
