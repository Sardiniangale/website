%% LyX 2.4.3 created this file.  For more info, see https://www.lyx.org/.
%% Do not edit unless you really know what you are doing.
\documentclass[american]{article}
\usepackage[T1]{fontenc}
\usepackage[utf8]{luainputenc}
\usepackage{amssymb}
\usepackage{stmaryrd}
\usepackage{babel}
\begin{document}
\title{Linear Algebra}
\date{April 08, 2025}
\author{Giacomo}
\maketitle

\part*{Pre-Det}

\section*{Spaces (Spazi Vettoriali)}

\textquotedbl I have not failed. I've just found 10,000 ways that
won't work.\textquotedbl{}

\subsection*{Basic Definitions}

\paragraph*{Subsets (Sottoinsiemi):}

This is a set where all the elements are contained within another
set. For linear algebra it could be a collection of vectors within
a vector space e.g $\mathbb{R}^{3}$

\[
S=\left\{ \left(x,y,0\right)|x,y\in\mathbb{R}\right\} 
\]


\paragraph*{Proper Subsets (Sottoinsiemi Propri):}

This is a subset that is always smaller than the original set. The
exact definition is: 

\begin{equation}
A\subsetneq B
\end{equation}
if:

1. Containment: ``Every element of A is also a element of B''

\begin{equation}
\forall x(x\in A\Longmapsto x\in B)
\end{equation}

2. Not equal: ``There exists at least one element in B that is not
A''

\begin{equation}
\exists y(y\in B\land y\notin A)
\end{equation}


\paragraph*{Supersets (Soprainsiemi):}

This is the opposite of a subset. If $A\subset B$, then B is a superset
of A

\subsection*{Linear Independence}

Let V be a vector space over a field F. $S\subseteq V$ is linearly
independent if the following conditions are true:

For a collection of vectors $\left\{ v_{1},v_{2},...,v_{n}\right\} \subseteq S$
and scalars $\left\{ a_{1},a_{2},...,a_{n}\right\} \in F$

\begin{equation}
a_{1}v_{1}+a_{2}v_{2}+...+a_{n}v_{n}=0
\end{equation}

i.e every single variant of a up to n has to equal zero. If it does
not, it is linearly dependent

\subsection*{Dimensions of a space}

The dimension (dimensione) of a vector space (spazio vettoriale) is
the number of vectors in any basis (base) for that space. A basis
(base) is a set of vectors that are linearly independent (linearmente
indipendenti) and span (generano) the entire space.

\subsection*{Sum of subspace (Somma di sottospazi)}

Let U and V be subspaces (sottospazi) of a vector space (spazio vettoriale)
V over a field $\mathbb{F}$. The sum (somma) can be defined as:

\begin{equation}
U+V=\left\{ u+v|u\in U,v\in V\right\} 
\end{equation}


\subsection*{Direct Sum (Somma diretta)}

U+V is called a direct sum, if the intersection of U and V is trivial
???- (Further desc required) then:

\[
U\cap V=\left\{ 0\right\} 
\]
 The direct sum can be denoted as $U\oplus V:$--to be expanded upon

\[
z=u+v
\]

For subspaces \( U_1, U_2, \ldots, U_n \subseteq V \), the sum \( U_1 + U_2 + \cdots + U_n \) is a \textit{direct sum} (somma diretta) if for each \( i \),
\[
U_i \cap \left( \sum_{j=1,\; j \neq i}^n U_j \right) = \{ 0 \}.
\]
The direct sum is denoted \( U_1 \oplus U_2 \oplus \cdots \oplus U_n \), and every \( z \in U_1 \oplus U_2 \oplus \cdots \oplus U_n \) has a unique representation:
\[
z = u_1 + u_2 + \cdots + u_n \quad \text{with } u_i \in U_i\ \forall i.
\]

\paragraph{Formal Uniqueness Condition:}
For \( U \oplus V \),
\[
\forall z \in U \oplus V,\ \exists!\, (u, v) \in U \times V \text{ such that } z = u + v.
\]
Here, \( \exists! \) denotes ``there exists exactly one.''

\part*{Post-Det}
\end{document}
