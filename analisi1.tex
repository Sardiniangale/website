%% LyX 2.4.3 created this file.  For more info, see https://www.lyx.org/.
%% Do not edit unless you really know what you are doing.
\documentclass[british]{article}
\renewcommand{\familydefault}{\sfdefault}
\usepackage[LGR,T1]{fontenc}
\usepackage[utf8]{luainputenc}
\usepackage{color}
\usepackage{mathtools}
\usepackage{amsmath}
\usepackage{amssymb}
\usepackage{graphicx}
\PassOptionsToPackage{normalem}{ulem}
\usepackage{ulem}

\makeatletter

%%%%%%%%%%%%%%%%%%%%%%%%%%%%%% LyX specific LaTeX commands.

\newcommand*\LyXbar{\rule[0.585ex]{1.2em}{0.25pt}}
\DeclareRobustCommand{\greektext}{%
  \fontencoding{LGR}\selectfont\def\encodingdefault{LGR}}
\DeclareRobustCommand{\textgreek}[1]{\leavevmode{\greektext #1}}

%% A simple dot to overcome graphicx limitations
\newcommand{\lyxdot}{.}


\makeatother

\usepackage{babel}
\begin{document}
\title{Mathematical Analisis 1}
\date{22 April 2025}
\author{Giacomo}

\maketitle
\includegraphics[scale=0.5]{/home/guc/Pictures/Cover}

\newpage{}

\part*{Pre-Derivatives}

\section*{Theorems, functions and axioms}

``Calvin: You know, I don’t think math is a science, I think it’s
a religion.

Hobbes: A religion?

Calvin: Yeah. All these equations are like miracles. You take two
numbers and when you add them, they magically become one NEW number!
No one can say how it happens. You either believe it or you don’t.
{[}Pointing at his math book{]} This whole book is full of things
that have to be accepted on faith! It’s a religion!''

\subsection*{Logic}

\subsubsection*{Propositional Logic}

A proposition is a statement. Statements in math are either true or
false, if you combined multiple propositions there are multiple outcomes
depending on the validity of each proposition in the statement. I
call them compound statements (but I dont think its the actual name).

\paragraph*{Negation (Negazione):}

Represented by a $\lnot$. Inverts the value of a statement, e.g (Bob
went to the store), the opposite can be represented as $\lnot$(Bob
went to the store).

\paragraph*{Disjunction (Disgiunzione):}

Represented by a $\vee$. Statement is true if at least one of the
propositions is also true.

\paragraph*{Conjunction (Congiunzione):}

Represented by a $\wedge.$ Statement is true only if both propositions
are also true.

\paragraph*{Implication (Implicazione):}

Represented by a $\rightarrow$. Its formally defined as : $\neg P\vee Q$.

\paragraph*{Biconditional (Bicondizionale):}

Represented by a $\leftrightarrow$. Statement is true if both of
the propositions share the same value.

\subsubsection*{Predicate Logic}

\paragraph*{Universal:}

It is represented by $\forall$. ``For every ...''

\paragraph*{Existence:}

It is represented by $\exists$. ``There exists ...''

\subsubsection*{Axioms}

\paragraph*{Axioms (Assiomi):}

An axiom is a postulate, more commonly known as assumption. It is
a statement that is held as always true in regards to the problem
or proof needed to solve. There are different types of axioms which
are briefly stated below:

\subparagraph*{Logical Axioms:}

A universal truth in all of Mathematics applicable in both the Physics
Notes and the Linear Algebra Notes.

\subparagraph*{Non Logical Axioms:}

Domain specific assumptions, such as Axioms only applicable in $\mathbb{R}$(Gross
oversimplification)

\subsection*{Group Theory (Theoria degli insiemi)}

\subsubsection*{Definition of a Group}

A group is a set, that has the following requirements:

\paragraph*{Closure:}

For all $a,b\in G,a*b\in G$

\paragraph*{Associativity:}

$a,b,c\in G,(a*b)*c=a*(c*b)$

\paragraph*{Identity:}

Lets say e exists $e\in G$ such that $e*a=a*e=a$ for all $a\in G$ 

\paragraph*{Inverse:}

If there exists a a, such that $a\in G$ therefore there exists a
inverses

\subsection*{Proofs (Dimostrazione)}

In Math, there every theorem and formula needs to be able to be proven
in a proof. There are multiple types of proofs which are used to show
that a theorem and or formula is valid.

\paragraph*{Direct Proof (Dimostrazione Diretta).}

By using known definitions, axioms and theorems, a sequence of logical
steps can be used to directly demonstrate whether or not the statement
is correct.

\paragraph{Proof by Contradiction (Dimostrazione per Assurdo).}

Assume that a statement is false and connect it to a logical contradiction. 

\paragraph*{Induction (Induzione).}

Used for statements involving natural numbers

\subparagraph*{Base Case: }

Verify the statement holds for the initial value

\subparagraph*{Inductive Step: }

Assume it holds for a n=k, then prove it holds for n=k+1.

\paragraph*{Constructive Proof (Dimostrazione Costruttiva).}

Make a identity with the exact desired property. More formally `'Demonstrates
the existence of an object by explicitly constructing it'\/'

\subsection*{Functions}

A \textbf{function} (\textit{funzione}) is a very common conecept in math that formalizes the relationship between two sets by assigning each element of the first set to exactly one element of the second set. Formally, a function \( f: A \to B \) consists of:
\begin{itemize}
    \item A \textbf{domain} (\textit{dominio}) \( A \), the set of all possible inputs.
    \item A \textbf{codomain} (\textit{codominio}) \( B \), the set into which all outputs are mapped.
    \item A rule or correspondence that links each element \( x \in A \) to a unique element \( f(x) \in B \).
\end{itemize}

\paragraph*{Formal Definition}
A function \( f \) is a subset of the Cartesian product \( A \times B \) such that for every \( x \in A \), there exists exactly one \( y \in B \) where \( (x, y) \in f \). This is denoted as \( y = f(x) \).

\paragraph*{Key Properties}
\begin{enumerate}
    \item \textbf{Injectivity} (\textit{iniettiva}): A function is injective if distinct inputs map to distinct outputs:
    \[
        \forall x_1, x_2 \in A, \quad f(x_1) = f(x_2) \implies x_1 = x_2.
    \]
    \item \textbf{Surjectivity} (\textit{suriettiva}): A function is surjective if every element in \( B \) is an output for some input:
    \[
        \forall y \in B, \quad \exists x \in A \text{ such that } y = f(x).
    \]
    \item \textbf{Bijectivity} (\textit{biiettiva}): A function is bijective if it is both injective and surjective, establishing a one-to-one correspondence between \( A \) and \( B \).
\end{enumerate}

\subsection*{Natural Numbers}



The \textbf{natural numbers} (\textit{numeri naturali}) are the standard version of numbers in maths, used for counting and ordering. Formally, the set of natural numbers \( \mathbb{N} \) is defined as:
\[
\mathbb{N} = \{1, 2, 3, \ldots\} \quad \text{(sometimes including } 0 \text{ depending on context)}.
\]
They are characterized by their discrete, non-negative integer values and form the foundation for number theory and arithmetic.

\paragraph*{Formal Definition (Peano Axioms)}
The properties of natural numbers are axiomatically defined by the \textbf{Peano axioms} (\textit{assiomi di Peano}):
\begin{enumerate}
    \item \( 1 \) (or \( 0 \)) is a natural number.
    \item Every natural number \( n \) has a unique successor \( S(n) \), which is also a natural number.
    \item \( 1 \) (or \( 0 \)) is not the successor of any natural number.
    \item Distinct natural numbers have distinct successors: \( S(m) = S(n) \implies m = n \).
    \item \textbf{Induction}: If a property holds for \( 1 \) (or \( 0 \)) and holds for \( S(n) \) whenever it holds for \( n \), then it holds for all natural numbers.
\end{enumerate}

\paragraph*{Key Properties}
\begin{itemize}
    \item \textbf{Closure under addition and multiplication}: For all \( a, b \in \mathbb{N} \), \( a + b \in \mathbb{N} \) and \( a \cdot b \in \mathbb{N} \).
    \item \textbf{Non-closure under subtraction and division}: Subtraction \( a - b \) or division \( a / b \) may not result in a natural number.
    \item \textbf{Well-ordering principle} (\textit{principio del buon ordinamento}): Every non-empty subset of \( \mathbb{N} \) has a least element.
    \item \textbf{Infinite cardinality}: \( \mathbb{N} \) is countably infinite.
\end{itemize}

\paragraph*{Number Theory}
Natural numbers are central to number theory, which studies:
\begin{enumerate}
    \item \textbf{Prime numbers} (\textit{numeri primi}): Natural numbers \( > 1 \) with no divisors other than \( 1 \) and themselves:
    \[
        \mathbb{P} = \{2, 3, 5, 7, 11, \ldots\}.
    \]
    \item \textbf{Divisibility}: A number \( a \) divides \( b \) (\( a \mid b \)) if \( \exists k \in \mathbb{N} \) such that \( b = a \cdot k \).
    \item \textbf{Mathematical induction} (\textit{induzione matematica}): A proof technique leveraging the Peano axioms.
    \item \textbf{Modular arithmetic} (\textit{aritmetica modulare}): Operations on residues modulo \( n \), e.g., \( 7 \equiv 2 \mod 5 \).
\end{enumerate}


\subsection*{Whole Numbers}

\textbf{Whole Numbers} (\textit{numeri interi non negativi}) are an extension of the natural numbers that include zero, forming the set \( \mathbb{W} = \{0, 1, 2, 3, \ldots\} \). They are used for counting discrete objects and represent non-negative integers without fractions or decimals. 

\paragraph*{\textbf{Formal Definition}}  
The set \( \mathbb{W} \) satisfies the \textbf{Peano axioms} (\textit{assiomi di Peano}) with zero as the base element:
\begin{itemize}
    \item \( 0 \) is a whole number.
    \item Every whole number \( n \) has a unique successor \( S(n) \in \mathbb{W} \).
    \item \( 0 \) is not the successor of any whole number.
    \item Distinct numbers have distinct successors: \( S(a) = S(b) \implies a = b \).
    \item \textbf{Induction}: If a property holds for \( 0 \) and for \( S(n) \) whenever it holds for \( n \), it holds for all \( \mathbb{W} \).
\end{itemize}

\paragraph*{\textbf{Key Properties}}  
\begin{itemize}
    \item \textbf{Closure under addition and multiplication}: For \( a, b \in \mathbb{W} \), \( a + b \in \mathbb{W} \) and \( a \cdot b \in \mathbb{W} \).
    \item \textbf{Non-closure under subtraction}: \( a - b \in \mathbb{W} \) only if \( a \geq b \).
    \item \textbf{Additive identity}: \( 0 + a = a \) for all \( a \in \mathbb{W} \).
    \item \textbf{Well-ordering principle} (\textit{principio del buon ordinamento}): Every non-empty subset of \( \mathbb{W} \) has a least element.
\end{itemize}

\paragraph*{\textbf{Representation}}  
Whole numbers are represented in numeral systems such as:
\begin{itemize}
    \item \textbf{Decimal}: \( 0, 1, 2, \ldots \)
    \item \textbf{Binary}: \( 0_2 = 0_{10}, 1_2 = 1_{10}, 10_2 = 2_{10} \)
    \item \textbf{Unary}: \( 0 \) (often represented as an absence of marks), \( | = 1, || = 2 \).
\end{itemize}

\paragraph*{\textbf{Differences from Natural Numbers}}  
Unlike natural numbers (\textit{numeri naturali}), which sometimes exclude zero, whole numbers explicitly include \( 0 \). This makes \( \mathbb{W} \) the set \( \mathbb{N} \cup \{0\} \) in contexts where natural numbers start at \( 1 \).


\paragraph*{Below,}

is a image of all the relevant groups of numbers and how they are
related. Not all of them have been stated in this point in the notes,
but they are all relevant for analysis 1

\includegraphics[scale=0.2]{/home/guc/Pictures/NumberSetinC\lyxdot svg}

\subsection*{Trigonometric Functions}

\textbf{Trigonometric Functions} (\textit{funzioni trigonometriche}) are periodic functions that relate angles in a right triangle or on the unit circle to ratios of side lengths. The primary trigonometric functions are sine (\(\sin\)), cosine (\(\cos\)), tangent (\(\tan\)), and their reciprocals: cosecant (\(\csc\)), secant (\(\sec\)), and cotangent (\(\cot\)).

\paragraph*{\textbf{Formal Definition (Unit Circle)}}  
For an angle \(\theta\) measured counterclockwise from the positive \(x\)-axis on the unit circle (\(x^2 + y^2 = 1\)):
\[
\sin\theta = y, \quad \cos\theta = x, \quad \tan\theta = \frac{y}{x} \quad (x \neq 0).
\]
The reciprocals are defined as:
\[
\csc\theta = \frac{1}{\sin\theta}, \quad \sec\theta = \frac{1}{\cos\theta}, \quad \cot\theta = \frac{x}{y} \quad (y \neq 0).
\]

\paragraph*{\textbf{Key Properties}}  
\begin{itemize}
    \item \textbf{Periodicity} (\textit{periodicità}): \(\sin\theta\) and \(\cos\theta\) have period \(2\pi\); \(\tan\theta\) and \(\cot\theta\) have period \(\pi\).
    \item \textbf{Range}: 
    \[
    \sin\theta, \cos\theta \in [-1, 1]; \quad \tan\theta \in \mathbb{R} \text{ (excluding asymptotes)}.
    \]
    \item \textbf{Parity}: \(\sin\theta\) and \(\tan\theta\) are odd functions; \(\cos\theta\) is even:
    \[
    \sin(-\theta) = -\sin\theta, \quad \cos(-\theta) = \cos\theta.
    \]
    \item \textbf{Pythagorean Identity} (\textit{identità pitagorica}):
    \[
    \sin^2\theta + \cos^2\theta = 1.
    \]
\end{itemize}

\paragraph*{\textbf{Differences from Other Functions}}  
Unlike polynomial or exponential functions, trigonometric functions:
\begin{itemize}
    \item Are periodic and bounded (except \(\tan\theta\) and \(\cot\theta\)).
    \item Model oscillatory behavior (e.g., waves, circular motion).
    \item Require angular input (radians or degrees) rather than purely scalar quantities.
\end{itemize}

\paragraph*{Unit Circle For Reference:}

\includegraphics[scale=0.1]{/home/guc/Downloads/2048px-Unit_circle_angles\lyxdot svg}

\subsection*{Complex Numbers introduction}

The largest domain covered in these notes: $\mathbb{C}$. Complex
numbers are an extension of real numbers, defined by the imaginary
number i With this strange property where $i^{2}=-1$. They are used
to resolve polynomial equations unsolvable in real numbers, as shown
in the fundamental theorem of algebra (Might need a section on this).
Below is a complex number with iy as the imaginary component and x
as the real component.

\begin{equation}
z=x+iy
\end{equation}
 Imagine $\mathbb{R}$ covering the whole x axis, and $\mathbb{C}$
covering the whole y axis, that's the complex plane.

\subsubsection*{Operations}

Addition

\begin{equation}
z_{1}+z_{2}=(x_{1}+x_{2})+i(y_{1}+y_{2})
\end{equation}
 Subtraction

\begin{equation}
z_{1}-z_{2}=(x_{1}-x_{2})+i(y_{1}-y_{2})
\end{equation}
 Multiplication

\begin{equation}
z_{1}z_{2}=(x_{1}+iy_{1})(x_{2}+iy_{2})=(x_{1}x_{2}-y_{1}y_{2})+i(x_{1}y_{2}+x_{2}y_{1})
\end{equation}
 Complex Conjugate

\begin{equation}
\bar{z}=x-iy
\end{equation}
 Modulus

\begin{equation}
|z|=\sqrt{x^{2}+y^{2}}
\end{equation}
 Inverse

\begin{equation}
z^{-1}=\frac{\bar{z}}{|z|^{2}}=\frac{x-iy}{x^{2}+y^{2}}
\end{equation}

A image of the $\mathbb{C}$ plane for reference, showing the inverse
modulus as well:

\includegraphics[scale=0.2]{/home/guc/Pictures/Complex_conjugate_picture\lyxdot svg}

\subsection*{Group Theory}


\section*{Sums and Sequences}

\textquotedbl La situazione è grave ma non è seria.\textquotedbl{}

\subsection*{Limits}

\paragraph*{How its represented.}

Any sequence that converges to a limit is represented as:

\[
lim_{n\rightarrow\infty}a_{n}=L
\]

Where $\left\{ a_{n}\right\} $is the sequence.

\paragraph*{Formal Definition.}

\begin{equation*}
    \lim_{n \to \infty} a_n = L \quad \iff \quad 
    \forall \varepsilon > 0,\; \exists N \in \mathbb{N},\; \forall n \geq N,\; |a_n - L| < \varepsilon
\end{equation*}

``For every positive number, there exists a natural number such that,
for all integers, the distance between and L is less than \textgreek{ε}.''

\subsection*{Sequence}

\paragraph{Definition:}

A sequence/sucession is simply a list of $\mathbb{N}$ while following
a set of rules defined by a function. The function then maps each
$\mathbb{N}$ to a corresponding $\mathbb{R}$ following the rules
defined by the function. 

\paragraph*{Representation:}

It is often shown as a random letter (this case a) $a_{n}$ with n
representing the number of the term.

\subparagraph*{The first term, }

$a_{1}$ is called the initial term (termine iniziale)

\subparagraph*{The terms after the first,}

$a_{1+n}$ is called the recursive formula (formula ricorsiva)

\paragraph*{Types:}

There are 2 specific categories which will be covered in more detail.
1. Whether its bounded (limitata) or unbounded (illimitata). 2. Whether
its convergent (convergente), divergent (divergente) or oscillatory
(oscillante).

\subsection*{Bounded Successions (Successioni Limitate)}

\textcolor{red}{(The interval info in this subsection is subject to
get its own dedicated subsection)}

\paragraph*{Definition:}

A bounded sequence (successione limitata), is a sequence $a_{n}$
that exists within a range such that if $b\in\mathbb{R}$, b is greater
than $a_{n}$, and there exists $c\in\mathbb{R}$ that is less than
$a_{n}$, it is a bounded sequence. A bounded sequence is may suggest
convergences and can be proven by using the Bolzano-Weierstrass Theorem.

\paragraph*{Intervals:}

\subparagraph*{Open Interval}

\[
(a,b)\coloneqq\left\{ x\in\mathbb{R}|a<x<b\right\} 
\]

A open interval includes all $\mathbb{R}$ numbers between a and b,
excluding the endpoints. Represented by a () and <

\subparagraph*{Closed Interval }

\[
[a,b]\coloneqq\left\{ x\in\mathbb{R}|a\leq x\leq b\right\} 
\]

A closed interval includes all $\mathbb{R}$ numbers between a and
b, including the endpoints. Represented by a {[}{]} and $\leq$. 

\subparagraph*{Empty Interval}

Denoted as $\emptyset$, it contains no numbers.

\subparagraph*{Degenerate Interval}

A single point, $[a,a]=\left\{ a\right\} $. By technicality it is
always closed.

\subparagraph*{Half Intervals}

It is possible to have intervals which are open at one end and closed
at the other, and vice versa. e.g

\[
(a,b]\coloneqq\left\{ x\in\mathbb{R}|a<x\leq b\right\} 
\]


\subparagraph*{Infinite Intervals}

There are also intervals which are infinite on one side and open or
closed on the other. e.g

\[
(a,+\infty)\coloneqq\left\{ x\in\mathbb{R}|a<x\right\} 
\]

The infinite can be negative as well on the other side. (Not sure
if the infinite has to be in a open interval, because I have not seen
any which are not in a open interval)

\paragraph*{Types of bounds: }

\subparagraph*{Upper bound:}

If a upper bound exists we call it bounded from above.

\subparagraph*{Lower bound:}

If a lower bound exists we call it bounded from below.

If both upper and lower bound exist the set is bounded.

\subparagraph*{Bound properties:}

There can be multiple upper and lower bound (As clearly shown in the
diagram below). 

(THE UNDERLINE TEXT BELOW IS TO BE REMOVED due to no longer being
needed.)

\paragraph*{\textrm{\uline{Superiorly Limited:}}}

\textrm{\uline{If A Sequence In \mbox{$\mathbb{R}$} Diverges To \mbox{$-\infty$}
Then It Is Bounded Above (Superiormente Limitata)}}

\paragraph*{\textrm{\uline{Inferiorly Limited}}}

\textrm{\uline{If A Series In \mbox{$\mathbb{R}$} Diverges To \mbox{$+\infty$}
Then It Is Bounded Below (Inferiormente Limitata)}}

\paragraph*{Supremum and Infimum}

The supremum and infimum can only exist for a interval with at least
one open point. Its the smallest possible upper bound (If its a supremum)
or the largest possible lower bound (If its a infimum). It can NEVER
reach the interval. The Supermum can be written as $supM$ and the
Infium can be written as $infM$

\paragraph*{Minimum and Maximum}

For a Minimum or a Maximum you must have at least one closed interval
point. (as shown in the diagram below). The minimum or maximum is
the point a or b that hold the interval. 

\paragraph*{Diagram of open and closed sequences:}

\includegraphics[scale=0.4]{/home/guc/Pictures/boundedsequences}

\subsection*{Monotone Sequences }

\paragraph*{Definition:}

A sequence that is monotone is either non-decreasing or non-increasing.
Therefore, by extention a constant sequence is simultaneously non-decreasing
and non-increasing. Therefore it is monotone. The strictly increasing/decreasing
are simply subsets.

\paragraph*{Growth of Sequences:}

\subparagraph*{Non-decreasing sequence (successioni crescenti):}

Is a sequence that increases if each term is greater than or equal
to the previous term.

\subparagraph*{Non-increasing sequences (successioni decrescenti):}

Is a sequence that it decreases if each term is less than or equal
to the previous term.

\subparagraph*{Strictly increasing sequences (successioni strettamente crescenti):}

Is a sequence that strictly increasing if each term is strictly greater
than the previous term.

\subparagraph*{Strictly decreasing sequences (successioni strettamente decrescenti):}

Is a sequence that strictly decreasing if each term is strictly less
than the previous term.

\subsection*{Convergent Sequences}


\subsection*{Cauchy Sequences}

\subsection*{Napier's Constant (Costante di Nepero)}

This may also be known as e or exponential or exp. It has 3 separate
definitions: Limit, Series, and integral. However I will only do the
explanation for limit definition here. The integral explanation will
be done in the integral chapter.

\paragraph*{Limit Definition:}

\begin{equation}
e=lim_{n\rightarrow\infty}\left(1+\frac{1}{n}\right)^{n}
\end{equation}


\paragraph*{Analisis Properties (still figuring out what to do with this section):}

$e^{x}$ is continuous in $\mathbb{R}$ and it is also infinitely
differentiable.

....

\subparagraph*{Base of the Natural Logarithm} 
\( e \) is the base of the natural logarithm, denoted as \( \ln(x) \).

\subparagraph*{Approximate Value} 
\( e \approx 2.71828 \).

\subparagraph*{Irrational Number} 
\( e \) cannot be expressed as a ratio of two integers.

\subparagraph*{Transcendental Number} 
\( e \) is not a root of any non-zero polynomial with rational coefficients.

\subparagraph*{Limit Definition} 
\[
e = \lim_{n \to \infty} \left(1 + \frac{1}{n}\right)^n
\]

\subparagraph*{Euler's Identity} 
\[
e^{i\pi} + 1 = 0
\]
Connecting \( e \), imaginary numbers, and \( \pi \)

\subsection*{Handling Infinity}

\subsection*{Exact Successions}

\subsection*{Bolzano Vistras Theorem}

\subsection*{Theorem of Zero}

\subsection*{Trigonometric Functions and $\pi$}

\subsection*{E and Euler}

Below are some old section from a scrapped chapter. Will probably
be useful here 

\subsubsection*{Exponential Properties}
For $a > 0$, $b > 0$, and $x,y \in \mathbb{R}$:
\begin{itemize}
\renewcommand{\labelitemi}{}
\setlength\itemsep{-0.5em}
\setlength\parsep{-0.5em}
    \item $a^x a^y = a^{x+y}$
    \item $\frac{a^x}{a^y} = a^{x-y}$
    \item $(a^x)^y = a^{xy}$
    \item $a^0 = 1$
    \item $a^{-x} = \frac{1}{a^x}$
    \item $(ab)^x = a^x b^x$
    \item $\left(\frac{a}{b}\right)^x = \frac{a^x}{b^x}$
\end{itemize}

\subsubsection*{Def of Logarithms}
For $a > 0$ ($a \neq 1$), $b > 0$:
\begin{itemize}
\renewcommand{\labelitemi}{}
    \item $\log_a b = x \iff a^x = b$
    \item $\log_a 1 = 0$
    \item $\log_a a = 1$
    \item $a^{\log_a b} = b$
\end{itemize}

\subsubsection*{Logarithm Properties}
For $a > 0$ ($a \neq 1$), $x,y > 0$, $k \in \mathbb{R}$:
\begin{itemize}
\renewcommand{\labelitemi}{}
    \item $\log_a(xy) = \log_a x + \log_a y$
    \item $\log_a\left(\frac{x}{y}\right) = \log_a x - \log_a y$
    \item $\log_a(x^k) = k\log_a x$
    \item $\log_a a^x = x$
\end{itemize}

\subsubsection*{Change of Base}
For $a,b > 0$ ($a,b \neq 1$), $c > 0$:
\[
\log_a c = \frac{\log_b c}{\log_b a}
\]
Special case: $\log_a b = \frac{1}{\log_b a}$

\subsubsection*{Natural Exponential and Logarithm}
\begin{itemize}
\renewcommand{\labelitemi}{}
    \item $e^x = \exp(x)$
    \item $\ln x = \log_e x$
    \item $e^{\ln x} = x$ for $x > 0$
    \item $\ln(e^x) = x$ for $x \in \mathbb{R}$
\end{itemize}

\subsubsection*{Exponential-Logarithmic Equations}
Key solving techniques:
\begin{itemize}
\renewcommand{\labelitemi}{}
    \item If $a^x = a^y$ then $x = y$
    \item If $\log_a x = \log_a y$ then $x = y$
    \item To solve $a^{f(x)} = b$: take logarithms of both sides
    \item To solve $\log_a f(x) = b$: rewrite as $f(x) = a^b$
\end{itemize}

\subsection*{Complex Polynomials}

\section*{Series}

\newpage{}

\part*{Post-Derivatives}

\section*{Derivatives}

``The Difference Between the Almost Right Word and the Right Word
Is Really a Large Matter---’Tis the Difference Between the Lightning
Bug and the Lightning'' - Mark Twain

\subsubsection*{
\begin{equation}
f'(x)=lim_{h\rightarrow0}\frac{f(x+h)-f(x)}{h}
\end{equation}
}

\subsubsection*{Methods for Differentiation}

\paragraph*{Product rule}

\begin{equation}
\frac{d}{dx}\left[f(x)g(x)\right]=f'(x)g(x)+f(x)g'(x)
\end{equation}


\paragraph{Quotient Rule}

\begin{equation}
\frac{d}{dx}\left(\frac{f(x)}{g(x)}\right)=\frac{f'(x)g(x)-f(x)g'(x)}{g(x)^{2}}
\end{equation}


\paragraph*{Chain Rule}

\begin{equation}
\frac{d}{dx}f(g(x))=f'(g(x))g'(x)
\end{equation}


\section*{Integrals}

“If you're gonna shoot an elephant Mr. Schneider, you better be prepared
to finish the job.”

\LyXbar{} Gary Larson, The Far Side 

\subsubsection*{Riemann Integral}

A Riemann integral is the the limit $f:[a,b]\rightarrow\mathbb{R}$
of all the Riemann sums between the points a and b. Defined as:

\[
\int_{a}^{b}f(x)dx=lim_{||P||\rightarrow0}\sum_{i=1}^{n}f(c_{i})\Delta x_{i}
\]
 Where:

n is the sub-intervals

$\Delta x_{i}$is the width of the i sub-interval

$c_{i}$is a sample point

{*}this exact definition will likely never be used in 

\paragraph*{Properties,}

\begin{gather*}
    \text{Linearity:} \quad \int_a^b \left( \alpha f(x) + \beta g(x) \right) dx = \alpha \int_a^b f(x) \, dx + \beta \int_a^b g(x) \, dx \\
    \text{Additivity:} \quad \int_a^b f(x) \, dx = \int_a^c f(x) \, dx + \int_c^b f(x) \, dx \quad \text{for } c \in (a, b) \\
    \text{Monotonicity:} \quad f(x) \leq g(x) \ \forall x \in [a, b] \implies \int_a^b f(x) \, dx \leq \int_a^b g(x) \, dx \\
    \text{Bounds:} \quad m(b - a) \leq \int_a^b f(x) \, dx \leq M(b - a) \quad \text{where } m = \inf_{[a,b]} f,\ M = \sup_{[a,b]} f \\
    \text{Integrability:} \quad f \text{ is Riemann integrable } \iff f \text{ bounded } \land \ f \text{ continuous a.e. on } [a, b] \\
    \text{Fundamental Theorem:} \quad F' = f \implies \int_a^b f(x) \, dx = F(b) - F(a)
\end{gather*}

\subsubsection*{Fundamental theorem of calculus}

\subsubsection*{Methods for integration}

\paragraph{Substitution,}

if $u=f(x)$, then $du=g'(x)dx$

\begin{equation}
\int f(g(x))g'(x)dx=\int f(u)du
\end{equation}


\paragraph{Integration by parts}

\begin{equation}
\int udv=uv-\int vdu
\end{equation}


\paragraph{Product rule}

\begin{equation}
\end{equation}


\paragraph{Chain Rule}

\paragraph{Lebiz rule for differentiation.}

If f(t) is continuous and g(x) is differentiable
\[
f(x)=\int_{a}^{g(x)}f(t)dt
\]

If this is true, then:

\begin{equation}
F(x)'=f\left(g(x)\right)\cdot g'(x)
\end{equation}

If both the limits are dependent on x:

\[
I(x)=\int_{g_{1}(x)}^{g_{2}(x)}f(t)dt
\]

then:

\begin{equation}
I'(x)=f\left(g_{2}(x)\right)g_{2}'(x)-f\left(g_{1}(x)\right)g_{1}'(x)
\end{equation}


\paragraph{Limit under a derviative}

\section*{Differential Equations}

\textquotedbl Would you tell me, please, which way I ought to go
from here?\textquotedbl{}

\textquotedbl That depends a good deal on where you want to get to,\textquotedbl{}
said the Cat.

\subsubsection*{ODEs and PDEs}

Ordinary Differential Equations (ODEs) are a differential equation
which has a single variable. ODEs have a general form:

\begin{equation}
F\left(x,y,\frac{dy}{dx},\frac{d^{2}y}{dx^{2}},...,\frac{d^{n}y}{dx^{n}}\right)=0
\end{equation}

where 

- x independent

- y dependent

Partial Differential Equations (PDEs) are a Differential equation
which has multiple independent variables. Instead of using the standard
d, they use partial derivatives (\ensuremath{\partial}) to show the
change with respect for multiple variables. The general form is:

\begin{equation}
F\left(x,y,u,\frac{\partial u}{\partial x},\frac{\partial u}{\partial y},\frac{\partial^{2}u}{\partial x^{2}},....\right)=0
\end{equation}

where

- x,y independent

- u(x,y) dependent

Fundamentally image a ODE as a means to track a single car, while
PDE track all the traffic in the city.

\subsubsection*{Types of ODEs}

ODEs are usually classified by 2 primary things. their order, aka
the degree of their derivative, and by whether they are linear or
non-linear. 

\paragraph*{First order ODEs}

are pretty self explanatory, they involve only the 1st derivative.
Here is a basic first order ODE:

\paragraph*{
\begin{equation}
\frac{dy}{dx}+y=x
\end{equation}
Second order ODEs}

involve UP to the 2nd derivative. Here is a example:

\paragraph*{
\begin{equation}
\frac{d^{2}y}{dx^{2}}+2\frac{dy}{dx}+y=0
\end{equation}
Higher order ODEs}

involve everything 3rd derivative or higher. It is unlikely to ever
appear in a 1st year analisis exam, but you never know. 

\paragraph*{Liniar and Non-liniear ODEs.}

An ODE is linear if the dependent variable and the derivatives are
in a linear form. Basically: they are not multiplied together. Anything
else is considered non-liniear. A linear ODE can be written in the
form:

\begin{equation}
a_{n}(x)\frac{d^{n}y}{dx^{n}}+a_{n-1}(x)\frac{d^{n-1}y}{dx^{n-1}}+...+a_{1}(x)\frac{dy}{dx}+a_{0}(x)y=f(x)
\end{equation}

- a(x) is a function of x

Here are some basic examples. We will go in much more detail when
solving ODEs.

$\frac{dy}{dx}+3y=x$, and $\frac{d^{2}y}{dx^{2}}+x\frac{dy}{dx}+y=sinx$

\[
Ineedtolearnbetterformating:)
\]
 A non-linear ODE is any ordinary differential equation that cannot
be written in the linear form shown earlier. This is because the dependent
variable or its derivatives are not linear. I will cover this more
later on in the chapter.

\subsubsection*{The weird classifications of ODEs}

\paragraph{A Homogeneous ODE,}

is a differential equation where L is a linear differential operator.

\paragraph{
\[
L[y]=0
\]
 A Non-Homogeneous ODE,}

is a differential equation where f(x) \ensuremath{\neq} 0

\paragraph{
\[
L[y]=f(x)
\]
 A Autonomous ODE,}

is a differential equation where the independent variable (usually
x or t) does not appear in the equation.

\paragraph*{
\[
\frac{d^{n}y}{dx^{n}}=F\left(y,y',...,y^{(n-1)}\right)
\]
A Non-Autonomous ODE,}

is a differential equation if the independent variable appears eq
(1).

{*}You can use multiple types at once, just use common sense to make
sure its right

\subsubsection*{Basic existence}

\subsection*{Dealing with ODEs}

\paragraph*{Separable ODE,}

can be written as 
\[
\frac{dy}{dx}=f(x)g(y)
\]
 Divide both sides by g(y), and multiply both sides by dx
\[
\frac{dy}{g(y)}=f(x)dx
\]
 Integrate both sides, make sure to keep the constant on the RHS
\[
\int\frac{dy}{g(y)}=\int f(x)dx
\]


\paragraph*{Integrating Method.}

Format the equation to fit the following before using the method
\[
\frac{dy}{dx}+P(x)y=Q(x)
\]
 Find the function $\mu(x)$ that will help simplify the problem.
(Symplify as much as possible here it will help a lot later on)
\[
\mu(x)=e^{\int P(x)dx}
\]
Multiply every term by the function $\mu(x)$ and using the product
rule calculate the derivative of the LHS
\[
\frac{d}{dx}(\mu(x)y)=\mu(x)Q(x)
\]
 Integrate both sides, remember the constant!
\[
y=\frac{1}{\mu(x)}\int\mu(x)Q(x)dx+C
\]

Quick note on integrating factor. A integrating factor is the method
used to solve derivative equation above. The \textgreek{μ}(x) also
called the integrating factor, works for any, first-order linear differential
equations. A function is derived by multiplying the equation with
\textgreek{μ}(x), which makes the left-hand side a derivative of \textgreek{μ}(x)y.

\paragraph*{Exact Method}

If a DE is exact, which can be found if it is in this form

\[
M(x,y)dx+N(x,y)dy=0
\]
 After this, calculate the partial derivative of M in respect to y
and the partial derivative of N with respect to x. If these are equivalent
the DE is exact.

\[
\frac{\partial M}{\partial y}=\frac{\partial N}{\partial x}
\]

Lets make a function that we will call $\Psi$ such that, $\varPsi_{x}=M(x,y)$
and $\Psi_{y}=N(x,y)$

Therefore we can write this now as

\[
\Psi_{x}+\Psi_{y}\frac{dy}{dx}=0
\]
we can start to find this function $\Psi$. So we will start to integrate
M with respect to x. (h(y) is a function of y)

\[
\Psi(x,y)=\int Mdx+h(y)
\]
 We can now differentiate $\Psi$ with respect to y

\[
\frac{\partial\Psi}{\partial y}=\frac{\partial}{\partial y}\left(\int Mdx\right)+h'(y)=N
\]
 Solve the integral for h(y)

\subsection*{Types of Problems}

\subsubsection*{Initial Value Problem}

Generally, IVPs are a DE and a \uline{initial condition} or condition's
which when used in unison they can be used to solve a function, that
will also fit the DE. The steps are pretty straight forward. 

1. Solve the DE

\[
y(x)=\int f(x)dx+C
\]

2. Use the initial condition, lets say that $y(x_{0})=y_{0}$

\[
y_{0}=\int f(x_{0})dx+C
\]

Where C is:

\[
C=y_{0}-V
\]
 {*}V is the value of the integral at $x_{0}$, therefore if we replace
C, the final answer is

\paragraph*{
\[
y(x)=\protect\int f(x)dx+(y_{0}-V)
\]
}

\paragraph*{Proving existence and uniqueness}

\subparagraph*{Theorem Definition:}

If f(x,y) and the partial derivative $\frac{\partial f}{\partial y}$
are continuous in:

\[
D=\left\{ (x,y)||x-x_{0}|\leq a,|y-y_{0}|\leq b\right\} 
\]
around the point $(x_{0},y_{0})$ therefore, there exits a interval
between x and $x_{0}$where there is at least one solution of y(x).
And it proves that this solution is unique on the interval.

\subparagraph*{Steps:}

Write the ODE in standard form, aka:

\[
\frac{dy}{dx}=f(x,y)
\]

use intuition to check that the function f is continuous between the
points you want. Then if there are any points where the function is
non continuous, then be sure to mark it such that it is clear, using
$\leq\geq><$. This is the first rule.

Now, do a partial derivative of y such that:

\[
\frac{\partial f}{\partial y}=f(x,y)
\]

Remember to treat x as a constant in this case!!!

If the result is continuous in D between x and $x_{0}$ then this
proves that there is at least a solution. 

\subsection*{Interpreting answers}

\paragraph{Interval of validity for Linear DE.}

The interval of validity for a Linear DE is largest around $x_{0}$where
$p(x)$ and $q(x)$ are continuous. Make sure to exclude discontinuities.

\paragraph{Interval of validity for Non-Linear DE,}

Solutions may be exponential to infinity or become undefined despite
$f(x,y)$ being smooth. Therefore check all the points where the solution
becomes undefined. Extend the interval on both sides of $x_{0}$ till
it is no longer possible.

\subsubsection*{Interpreting answers}

(this whole section is a bit useless, I may remove it if I don't find
any use for it son)

\paragraph*{Explicit Solution,}

is when the dependent variable is isolated and in terms of the independent
variable. e.g

\[
y=x^{2}+C
\]

(soluzione esplicita)

\paragraph*{Implicit Solution,}

is when the dependent variable is not explicitly isolated from the
independent. e.g

\[
x^{2}+y^{2}=C
\]

(soluzione implicita)

\paragraph{General Solution,}

is the solution containing all the possible solutions for the differential
equation, ie it keeps the constants, its the trivial form. (soluzione
generale)

\paragraph*{Particular Solution,}

is the solution which is a specific solution by locking the constants
by using the initial conditions, ie the C has a fixed value. (soluzione
particolare)

\paragraph*{Equilibrium Solution,}

is a solution which is constant because the dependent variable does
not change and therefore the derivative is zero. (soluzione di equilibrio)

\paragraph*{Parametric Solution,}

is a solution represented using a parameter like (t,u,z..) instead
of using x and y. eg.

\[
y(t)=\sqrt{t^{2}+C}
\]

(soluzione parametrica)

\subsubsection*{Second Order Equations-- temporary latex code which isnt my own
for the exam}



\subsubsection*{Solving Second-Order Differential Equations}
A second-order differential equation has the general form:
\[
F(y'', y', y, x) = 0
\]
Below are methods for solving linear and nonlinear cases.

%-------------------------------------------------
\paragraph{1. Linear Homogeneous Equations with Constant Coefficients}
\textbf{General form:}
\[
a y'' + b y' + c y = 0 \quad (a \neq 0)
\]
\textbf{Solution procedure:}
\begin{enumerate}
    \item Solve the \textbf{characteristic equation}:
    \[
    a r^2 + b r + c = 0
    \]
    
    \item \textbf{Case analysis for roots \( r_1, r_2 \):}
    \begin{itemize}
        \item \textbf{Distinct real roots:}  
        If \( r_1 \neq r_2 \),
        \[
        y(x) = C_1 e^{r_1 x} + C_2 e^{r_2 x}
        \]
        
        \item \textbf{Repeated real root:}  
        If \( r_1 = r_2 = r \),
        \[
        y(x) = (C_1 + C_2 x)e^{r x}
        \]
        
        \item \textbf{Complex conjugate roots:}  
        If \( r = \alpha \pm i\beta \),
        \[
        y(x) = e^{\alpha x}\left[C_1 \cos(\beta x) + C_2 \sin(\beta x)\right]
        \]
    \end{itemize}
\end{enumerate}

%-------------------------------------------------
\paragraph{2. Linear Nonhomogeneous Equations}
\textbf{General form:}
\[
y'' + p(x)y' + q(x)y = g(x)
\]
\textbf{Method 1: Undetermined Coefficients}
\begin{enumerate}
    \item Find the complementary solution \( y_c \) (solve the homogeneous equation).
    \item Assume a particular solution \( y_p \) based on \( g(x) \) (e.g., \( g(x) = e^{kx} \Rightarrow y_p = Ae^{kx} \)).
    \item If \( g(x) \) matches part of \( y_c \), multiply \( y_p \) by \( x \) (or \( x^n \) for repeated roots).
    \item Substitute \( y_p \) into the DE and solve for coefficients.
    \item General solution: \( y = y_c + y_p \).
\end{enumerate}

\textbf{Method 2: Variation of Parameters}
\begin{enumerate}
    \item Find \( y_c = C_1 y_1 + C_2 y_2 \).
    \item Compute the Wronskian: 
    \[
    W = y_1 y_2' - y_1' y_2
    \]
    \item Particular solution:
    \[
    y_p = -y_1 \int \frac{y_2 g(x)}{W} \, dx + y_2 \int \frac{y_1 g(x)}{W} \, dx
    \]
    \item General solution: \( y = y_c + y_p \).
\end{enumerate}

%-------------------------------------------------
\paragraph{3. Cauchy-Euler Equations}
\textbf{General form:}
\[
x^2 y'' + b x y' + c y = 0
\]
\textbf{Solution steps:}
\begin{enumerate}
    \item Assume \( y = x^r \). Substitute to get:
    \[
    r^2 + (b - 1)r + c = 0
    \]
    \item Solve for \( r \). The solution mirrors constant-coefficient cases:
    \begin{itemize}
        \item Real distinct roots: \( y = C_1 x^{r_1} + C_2 x^{r_2} \)
        \item Repeated root: \( y = x^{r}(C_1 + C_2 \ln x) \)
        \item Complex roots \( r = \alpha \pm i\beta \):  
        \( y = x^{\alpha}\left[C_1 \cos(\beta \ln x) + C_2 \sin(\beta \ln x)\right] \)
    \end{itemize}
\end{enumerate}

%-------------------------------------------------
\paragraph{4. Reduction of Order}
\textbf{When one solution \( y_1 \) is known:}
\begin{enumerate}
    \item Let \( y_2 = v(x) y_1 \).
    \item Substitute \( y_2 \) into the DE and solve for \( v(x) \).
    \item The second solution is \( y_2 = y_1 \int \frac{e^{-\int p(x) \, dx}}{y_1^2} \, dx \).
\end{enumerate}

\textbf{Special cases:}
\begin{itemize}
    \item \textbf{Equation missing \( y \):} Let \( p = y' \), reducing to \( p' = f(x, p) \).
    \item \textbf{Equation missing \( x \):} Let \( p = y' \), then \( y'' = p \frac{dp}{dy} \), reducing to \( p \frac{dp}{dy} = f(y, p) \).
\end{itemize}

%-------------------------------------------------
\paragraph{5. Series Solutions Near Ordinary Points}
For \( P(x)y'' + Q(x)y' + R(x)y = 0 \) with ordinary point at \( x_0 \):
\begin{enumerate}
    \item Assume \( y = \sum_{n=0}^\infty a_n (x - x_0)^n \).
    \item Substitute into DE and equate coefficients of like powers.
    \item Derive a recurrence relation for \( a_n \).
\end{enumerate}

%-------------------------------------------------
\paragraph{6. Laplace Transform for Initial Value Problems}
\textbf{Procedure:}
\begin{enumerate}
    \item Take Laplace transform of the DE:
    \[
    \mathcal{L}\{y''\} + a\mathcal{L}\{y'\} + b\mathcal{L}\{y\} = \mathcal{L}\{g(x)\}
    \]
    \item Use:
    \[
    \mathcal{L}\{y'\} = sY(s) - y(0), \quad \mathcal{L}\{y''\} = s^2Y(s) - sy(0) - y'(0)
    \]
    \item Solve for \( Y(s) \), then compute \( y(x) = \mathcal{L}^{-1}\{Y(s)\} \).
\end{enumerate}

%-------------------------------------------------
\paragraph*{Nonlinear Second-Order DEs}
No universal method exists. Common approaches:
\begin{itemize}
    \item Substitutions to reduce order (e.g., \( p = y' \))
    \item Exact equations (identify integrable combinations)
    \item Numerical methods (e.g., Runge-Kutta)
\end{itemize}

Homogeneous Equations with constant coefficients have the general
form:
\[
y''+ay'+by=0
\]
 Non-homogeneous Equations: Method of Undetermined Coefficients
\[
Finished
\]


\subsection*{Nth order Differential Equation}

\subsubsection{Solving \( n \)-th Order Differential Equations}  
\paragraph{Linear Homogeneous Equations with Constant Coefficients}  
Consider the equation:  
\[  
a_n y^{(n)} + a_{n-1} y^{(n-1)} + \cdots + a_1 y' + a_0 y = 0  
\]  

\textbf{Solution Steps:}
\begin{enumerate}
    \item \textbf{Form characteristic equation}:
    \[
    a_n r^n + a_{n-1} r^{n-1} + \cdots + a_1 r + a_0 = 0
    \]
    
    \item \textbf{Find roots} \( r_1, r_2, \ldots, r_n \)
    
    \item \textbf{Construct general solution}:
    \begin{itemize}
        \item \textit{Distinct real roots}:
        \[
        y_h = \sum_{i=1}^n C_i e^{r_i x}
        \]
        
        \item \textit{Repeated real root \( r \) with multiplicity \( k \)}:
        \[
        e^{r x}\left(C_1 + C_2 x + \cdots + C_k x^{k-1}\right)
        \]
        
        \item \textit{Complex conjugate pairs \( \alpha \pm \beta i \)}:
        \[
        e^{\alpha x}\left[C_1 \cos(\beta x) + C_2 \sin(\beta x)\right]
        \]
        For repeated pairs (multiplicity \( m \)):
        \[
        x^{m-1}e^{\alpha x}\left[C_1 \cos(\beta x) + C_2 \sin(\beta x)\right]
        \]
    \end{itemize}
\end{enumerate}

\paragraph{Linear Nonhomogeneous Equations}  
For equations:
\[
a_n y^{(n)} + \cdots + a_1 y' + a_0 y = g(x)
\]
\textbf{General Solution}: 
\[
y = y_h + y_p
\]
where \( y_h \) = homogeneous solution, \( y_p \) = particular solution.

\subparagraph{Method of Undetermined Coefficients}  
Use when \( g(x) \) is polynomial, exponential, sine, cosine, or combinations:

\begin{enumerate}
    \item Assume \( y_p \) with same form as \( g(x) \)
    \item If any term matches \( y_h \), multiply by \( x^s \) (\( s \) = smallest integer eliminating duplication)
    \item Substitute \( y_p \) into DE and solve for coefficients
\end{enumerate}

\subparagraph{Variation of Parameters}  
General method for arbitrary \( g(x) \):

\begin{enumerate}
    \item Find fundamental set \( \{y_1, \ldots, y_n\} \) from \( y_h \)
    \item Compute Wronskian:
    \[
    W(y_1, \ldots, y_n) = \begin{vmatrix}
    y_1 & y_2 & \cdots & y_n \\
    y_1' & y_2' & \cdots & y_n' \\
    \vdots & \vdots & \ddots & \vdots \\
    y_1^{(n-1)} & y_2^{(n-1)} & \cdots & y_n^{(n-1)}
    \end{vmatrix}
    \]
    
    \item Find \( u_i' = \frac{W_i}{W} \) where \( W_i \) = Wronskian with \( i \)-th column replaced by \( \begin{bmatrix} 0 \\ \vdots \\ g(x) \end{bmatrix} \)
    
    \item Integrate to get \( u_i \), then:
    \[
    y_p = \sum_{i=1}^n u_i y_i
    \]
\end{enumerate}

\paragraph{Variable Coefficient Equations}  
For \( y^{(n)} + P_{n-1}(x)y^{(n-1)} + \cdots + P_0(x)y = Q(x) \):

\subparagraph{Reduction of Order}  
If solution \( y_1 \) is known, let:
\[
y = y_1 \int v(x) dx
\]
Substitute to reduce equation order by 1.

\subparagraph{Cauchy-Euler Equations}  
Form: \( x^n y^{(n)} + a_{n-1} x^{n-1} y^{(n-1)} + \cdots + a_0 y = 0 \)

\begin{enumerate}
    \item Assume solution \( y = x^m \)
    \item Substitute to get characteristic equation:
    \[
    m(m-1)\cdots(m-n+1) + \sum_{k=0}^{n-1} a_k m(m-1)\cdots(m-k+1) = 0
    \]
    \item Handle roots as with constant coefficient equations
\end{enumerate}

\paragraph{Nonlinear Equations}  
\subparagraph{Order Reduction Techniques}
\begin{itemize}
    \item \textit{Missing \( y \)}: Let \( v = y' \), reduces order by 1
    \item \textit{Missing \( x \)}: Let \( v = y' \), then \( y'' = v \frac{dv}{dy} \), reduces to 1st order in \( v \)
\end{itemize}

\subparagraph{Exact Equations}  
If equation can be written as:
\[
\frac{d}{dx}\left[\text{Lower order expression}\right] = 0
\]
Integrate successively to solve.

\subparagraph{Integrating Factors}  
Find \( \mu(x) \) or \( \mu(y) \) such that:
\[
\mu(x)M(x,y)dx + \mu(x)N(x,y)dy = 0
\]
becomes exact.

\subparagraph{Special Forms}  
\begin{itemize}
    \item \textit{Bernoulli}: \( y' + P(x)y = Q(x)y^n \), use \( z = y^{1-n} \)
    \item \textit{Riccati}: \( y' = P(x)y^2 + Q(x)y + R(x) \), use \( y = y_1 + \frac{1}{v} \) if particular solution \( y_1 \) known
\end{itemize}

\subsection*{Systems of Differential Equations}

\subsubsection*{Solving Systems of Differential Equations}
\paragraph*{Linear Systems with Constant Coefficients}
For the system $\mathbf{x}' = A\mathbf{x}$ where $A$ is an $n \times n$ constant matrix:

\textbf{Solution Method:}
\begin{enumerate}
    \item \textbf{Find eigenvalues} $\lambda$ by solving:
    \[
    \det(A - \lambda I) = 0
    \]
    
    \item \textbf{Find eigenvectors} $\xi$ for each eigenvalue by solving:
    \[
    (A - \lambda I)\xi = 0
    \]
    
    \item \textbf{Construct general solution}:
    \begin{itemize}
        \item \textit{Real distinct eigenvalues}:
        \[
        \mathbf{x}(t) = \sum_{i=1}^n C_i e^{\lambda_i t} \xi_i
        \]
        
        \item \textit{Complex eigenvalues} $\alpha \pm \beta i$:
        \[
        \mathbf{x}(t) = C_1 e^{\alpha t}[\mathbf{a}\cos(\beta t) - \mathbf{b}\sin(\beta t)] + C_2 e^{\alpha t}[\mathbf{a}\sin(\beta t) + \mathbf{b}\cos(\beta t)]
        \]
        where $\mathbf{a} + i\mathbf{b}$ is the complex eigenvector
        
        \item \textit{Repeated eigenvalues}:
        \begin{itemize}
            \item If geometric multiplicity = algebraic multiplicity: proceed as distinct eigenvalues
            \item If deficient eigenvectors: use generalized eigenvectors
        \end{itemize}
    \end{itemize}
\end{enumerate}

\paragraph*{Nonhomogeneous Systems}
For $\mathbf{x}' = A\mathbf{x} + \mathbf{g}(t)$:

\textbf{General Solution}:
\[
\mathbf{x}(t) = \mathbf{x}_h(t) + \mathbf{x}_p(t)
\]

\subparagraph*{Variation of Parameters}
\begin{enumerate}
    \item Find fundamental matrix $\Phi(t)$ from homogeneous solutions
    \item Compute particular solution:
    \[
    \mathbf{x}_p(t) = \Phi(t) \int \Phi^{-1}(t)\mathbf{g}(t) dt
    \]
\end{enumerate}

\subparagraph*{Method of Undetermined Coefficients}
Use when $\mathbf{g}(t)$ contains polynomials, exponentials, or trigonometric functions:
\begin{enumerate}
    \item Assume $\mathbf{x}_p$ with same form as $\mathbf{g}(t)$
    \item Adjust for resonance if any term matches homogeneous solution
    \item Substitute and solve for coefficients
\end{enumerate}

\paragraph*{Matrix Exponential Method}
For $\mathbf{x}' = A\mathbf{x}$:
\[
\mathbf{x}(t) = e^{At}\mathbf{x}_0
\]
where $e^{At}$ can be computed via:
\begin{itemize}
    \item Diagonalization: $A = PDP^{-1} \Rightarrow e^{At} = Pe^{Dt}P^{-1}$
    \item Jordan form for defective matrices
    \item Taylor series expansion for simple cases
\end{itemize}

\paragraph*{Nonlinear Systems}
\subparagraph*{Linearization Near Critical Points}
\begin{enumerate}
    \item Find equilibrium points $\mathbf{x}_0$ where $\mathbf{f}(\mathbf{x}_0) = 0$
    \item Compute Jacobian matrix:
    \[
    J = \begin{bmatrix}
    \frac{\partial f_1}{\partial x_1} & \cdots & \frac{\partial f_1}{\partial x_n} \\
    \vdots & \ddots & \vdots \\
    \frac{\partial f_n}{\partial x_1} & \cdots & \frac{\partial f_n}{\partial x_n}
    \end{bmatrix}_{\mathbf{x}_0}
    \]
    \item Analyze eigenvalues of $J$ to determine stability
\end{enumerate}

\subparagraph*{Phase Plane Analysis (2D Systems)}
\begin{itemize}
    \item Classify critical points: node, spiral, saddle, center
    \item Use nullclines and direction fields
    \item Lyapunov functions for stability (when applicable)
\end{itemize}

\paragraph*{Conversion to First-Order Systems}
Any $n$-th order DE can be converted to a system:
\begin{enumerate}
    \item Let $x_1 = y$, $x_2 = y'$, ..., $x_n = y^{(n-1)}$
    \item Create system:
    \[
    \begin{cases}
        x_1' = x_2 \\
        x_2' = x_3 \\
        \vdots \\
        x_n' = F(t,x_1,\ldots,x_n)
    \end{cases}
    \]
\end{enumerate}

\paragraph*{Important Special Cases}
\subparagraph*{Coupled Oscillators}
\[
\begin{cases}
    m_1 x_1'' = -k_1 x_1 + k_2(x_2 - x_1) \\
    m_2 x_2'' = -k_2(x_2 - x_1)
\end{cases}
\]
Solve by diagonalizing the coefficient matrix

\subparagraph*{Competing Species Model}
\[
\begin{cases}
    x' = x(a - by) \\
    y' = y(c - dx)
\end{cases}
\]
Analyze using linearization and phase plane methods

\subsection*{Special theorems and Problems}

\subsubsection*{Picard--Lindelöf theorem}

\subsubsection*{Picard–Lindelöf Theorem (Existence \& Uniqueness)}
\paragraph*{Theorem Statement}  
Consider the initial value problem (IVP):
\[
\begin{cases}
    y'(t) = f(t, y(t)) \\
    y(t_0) = y_0
\end{cases}
\]
where \( f: D \subset \mathbb{R} \times \mathbb{R}^n \to \mathbb{R}^n \). If:

\begin{itemize}
    \item \( f \) is \textbf{continuous} in \( t \) on rectangle \( R = [t_0 - a, t_0 + a] \times \overline{B}(y_0, b) \)
    \item \( f \) is \textbf{Lipschitz continuous} in \( y \): 
    \[
    \exists L > 0 \text{ s.t. } \|f(t,y_1) - f(t,y_2)\| \leq L\|y_1 - y_2\|,\; \forall (t,y_1),(t,y_2) \in R
    \]
\end{itemize}

Then \( \exists \tau > 0 \) such that the IVP has a \textbf{unique solution} \( y(t) \) on \( [t_0 - \tau, t_0 + \tau] \).

\paragraph*{Proof Outline (Method of Successive Approximations)}  
\begin{enumerate}
    \item Reformulate IVP as integral equation:
    \[
    y(t) = y_0 + \int_{t_0}^t f(s, y(s)) ds
    \]
    
    \item Define Picard iterations:
    \[
    y_{n+1}(t) = y_0 + \int_{t_0}^t f(s, y_n(s)) ds
    \]
    Starting with \( y_0(t) \equiv y_0 \)
    
    \item Show \( \{y_n\} \) converges uniformly to solution \( y \):
    \begin{itemize}
        \item Use Lipschitz condition to prove \( \|y_{n+1} - y_n\| \leq \frac{M}{L} \frac{(L|t-t_0|)^{n+1}}{(n+1)!} \)
        \item Apply Banach fixed-point theorem in complete metric space
    \end{itemize}
\end{enumerate}

\paragraph*{Implementation Steps}  
To apply the theorem:
\begin{enumerate}
    \item Verify continuity of \( f(t,y) \) in \( t \)
    \item Check Lipschitz condition in \( y \):
    \begin{itemize}
        \item If \( \frac{\partial f}{\partial y} \) exists and bounded \( \Rightarrow \) Lipschitz
        \item For scalar case: \( |f(t,y_1) - f(t,y_2)| \leq L|y_1 - y_2| \)
    \end{itemize}
    
    \item Determine existence interval \( \tau = \min\left(a, \frac{b}{M}\right) \) where:
    \[
    M = \max_{(t,y)\in R} \|f(t,y)\|
    \]
\end{enumerate}

\paragraph*{Example Application}  
For IVP \( y' = y,\; y(0) = 1 \):
\begin{itemize}
    \item Picard iterations:
    \begin{align*}
        y_0(t) &= 1 \\
        y_1(t) &= 1 + \int_0^t y_0(s) ds = 1 + t \\
        y_2(t) &= 1 + \int_0^t (1+s) ds = 1 + t + \frac{t^2}{2} \\
        &\vdots \\
        y_n(t) &= \sum_{k=0}^n \frac{t^k}{k!} \to e^t
    \end{align*}
\end{itemize}

\paragraph*{Important Notes}  
\begin{itemize}
    \item \textbf{Local vs Global}: Theorem guarantees local solution - need additional conditions for global existence
    \item \textbf{Sharpness}: \( \tau \) estimate often conservative
    \item \textbf{Failure Cases}:
    \begin{itemize}
        \item \( f \) not Lipschitz \( \Rightarrow \) possible non-uniqueness (e.g., \( y' = \sqrt{|y|} \))
        \item Discontinuous \( f \) \( \Rightarrow \) solutions may not exist
    \end{itemize}
\end{itemize}

\subsubsection*{Cauchy's Problem}

Here we have a nth order ODE

\[
y^{(n)}(t)=f\left(t,y,y',....,y^{(n-1)}\right)
\]
 The Cauchy problem also known as the 
\end{document}
