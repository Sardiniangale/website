%% LyX 2.4.3 created this file.  For more info, see https://www.lyx.org/.
%% Do not edit unless you really know what you are doing.
\documentclass[american]{article}
\usepackage[LGR,T1]{fontenc}
\usepackage[utf8]{luainputenc}
\usepackage{amsmath}
\usepackage{amssymb}
\PassOptionsToPackage{normalem}{ulem}
\usepackage{ulem}

\makeatletter

%%%%%%%%%%%%%%%%%%%%%%%%%%%%%% LyX specific LaTeX commands.

\newcommand*\LyXbar{\rule[0.585ex]{1.2em}{0.25pt}}
\DeclareRobustCommand{\greektext}{%
  \fontencoding{LGR}\selectfont\def\encodingdefault{LGR}}
\DeclareRobustCommand{\textgreek}[1]{\leavevmode{\greektext #1}}


\makeatother

\usepackage{babel}
\begin{document}
\title{Analisi 1}
\date{March 27, 2025}
\author{Giacomo}
\maketitle

\part*{Pre-Derivatives}

\section*{Theorems, functions and axioms}

``Calvin: You know, I don’t think math is a science, I think it’s
a religion.

Hobbes: A religion?

Calvin: Yeah. All these equations are like miracles. You take two
numbers and when you add them, they magically become one NEW number!
No one can say how it happens. You either believe it or you don’t.
{[}Pointing at his math book{]} This whole book is full of things
that have to be accepted on faith! It’s a religion!''

\section*{Complex numbers}

A complex number is defined, by $x,y\in\mathbb{R}$ and i as the imaginary
unit.

\begin{equation}
z=x+iy
\end{equation}
 Imagine $\mathbb{R}$ covering the whole x axis, and $\mathbb{C}$
covering the whole y axis, that's the complex plane.

\subsubsection*{Operations}

Addition

\begin{equation}
z_{1}+z_{2}=(x_{1}+x_{2})+i(y_{1}+y_{2})
\end{equation}
 Subtraction

\begin{equation}
z_{1}-z_{2}=(x_{1}-x_{2})+i(y_{1}-y_{2})
\end{equation}
 Multiplication

\begin{equation}
z_{1}z_{2}=(x_{1}+iy_{1})(x_{2}+iy_{2})=(x_{1}x_{2}-y_{1}y_{2})+i(x_{1}y_{2}+x_{2}y_{1})
\end{equation}
 Complex Conjugate

\begin{equation}
\bar{z}=x-iy
\end{equation}
 Modulus

\begin{equation}
|z|=\sqrt{x^{2}+y^{2}}
\end{equation}
 Inverse

\begin{equation}
z^{-1}=\frac{\bar{z}}{|z|^{2}}=\frac{x-iy}{x^{2}+y^{2}}
\end{equation}


\section*{Sums and Sequences}

\section*{Series}

\part*{Post-Derivatives}

\section*{Derivatives}

``The Difference Between the Almost Right Word and the Right Word
Is Really a Large Matter---’Tis the Difference Between the Lightning
Bug and the Lightning'' - Mark Twain

\subsubsection*{
\begin{equation}
f'(x)=lim_{h\rightarrow0}\frac{f(x+h)-f(x)}{h}
\end{equation}
}

\subsubsection*{Methods for Differentiation}

\paragraph*{Product rule}

\begin{equation}
\frac{d}{dx}\left[f(x)g(x)\right]=f'(x)g(x)+f(x)g'(x)
\end{equation}


\paragraph{Quotient Rule}

\begin{equation}
\frac{d}{dx}\left(\frac{f(x)}{g(x)}\right)=\frac{f'(x)g(x)-f(x)g'(x)}{g(x)^{2}}
\end{equation}


\paragraph*{Chain Rule}

\begin{equation}
\frac{d}{dx}f(g(x))=f'(g(x))g'(x)
\end{equation}


\section*{Integrals}

“If you're gonna shoot an elephant Mr. Schneider, you better be prepared
to finish the job.”

\LyXbar{} Gary Larson, The Far Side 

\subsubsection*{Riemann Integral}

A Riemann integral is the the limit $f:[a,b]\rightarrow\mathbb{R}$
of all the Riemann sums between the points a and b. Defined as:

\[
\int_{a}^{b}f(x)dx=lim_{||P||\rightarrow0}\sum_{i=1}^{n}f(c_{i})\Delta x_{i}
\]
 Where:

n is the sub-intervals

$\Delta x_{i}$is the width of the i sub-interval

$c_{i}$is a sample point

{*}this exact definition will likely never be used in 

\paragraph*{Properties,}

\begin{gather*}
    \text{Linearity:} \quad \int_a^b \left( \alpha f(x) + \beta g(x) \right) dx = \alpha \int_a^b f(x) \, dx + \beta \int_a^b g(x) \, dx \\
    \text{Additivity:} \quad \int_a^b f(x) \, dx = \int_a^c f(x) \, dx + \int_c^b f(x) \, dx \quad \text{for } c \in (a, b) \\
    \text{Monotonicity:} \quad f(x) \leq g(x) \ \forall x \in [a, b] \implies \int_a^b f(x) \, dx \leq \int_a^b g(x) \, dx \\
    \text{Bounds:} \quad m(b - a) \leq \int_a^b f(x) \, dx \leq M(b - a) \quad \text{where } m = \inf_{[a,b]} f,\ M = \sup_{[a,b]} f \\
    \text{Integrability:} \quad f \text{ is Riemann integrable } \iff f \text{ bounded } \land \ f \text{ continuous a.e. on } [a, b] \\
    \text{Fundamental Theorem:} \quad F' = f \implies \int_a^b f(x) \, dx = F(b) - F(a)
\end{gather*}

\subsubsection*{Fundamental theorem of calculus}

\subsubsection*{Methods for integration}

\paragraph{Substitution,}

if $u=f(x)$, then $du=g'(x)dx$

\begin{equation}
\int f(g(x))g'(x)dx=\int f(u)du
\end{equation}


\paragraph{Integration by parts}

\begin{equation}
\int udv=uv-\int vdu
\end{equation}


\paragraph{Product rule}

\begin{equation}
\end{equation}


\paragraph{Chain Rule}

\paragraph{Lebiz rule for differentiation.}

If f(t) is continuous and g(x) is differentiable
\[
f(x)=\int_{a}^{g(x)}f(t)dt
\]

If this is true, then:

\begin{equation}
F(x)'=f\left(g(x)\right)\cdot g'(x)
\end{equation}

If both the limits are dependent on x:

\[
I(x)=\int_{g_{1}(x)}^{g_{2}(x)}f(t)dt
\]

then:

\begin{equation}
I'(x)=f\left(g_{2}(x)\right)g_{2}'(x)-f\left(g_{1}(x)\right)g_{1}'(x)
\end{equation}


\paragraph{Limit under a derviative}

\section*{Differential Equations}

\textquotedbl Would you tell me, please, which way I ought to go
from here?\textquotedbl{}

\textquotedbl That depends a good deal on where you want to get to,\textquotedbl{}
said the Cat.

\subsubsection*{ODEs and PDEs}

Ordinary Differential Equations (ODEs) are a differential equation
which has a single variable. ODEs have a general form:

\begin{equation}
F\left(x,y,\frac{dy}{dx},\frac{d^{2}y}{dx^{2}},...,\frac{d^{n}y}{dx^{n}}\right)=0
\end{equation}

where 

- x independent

- y dependent

Partial Differential Equations (PDEs) are a Differential equation
which has multiple independent variables. Instead of using the standard
d, they use partial derivatives (\ensuremath{\partial}) to show the
change with respect for multiple variables. The general form is:

\begin{equation}
F\left(x,y,u,\frac{\partial u}{\partial x},\frac{\partial u}{\partial y},\frac{\partial^{2}u}{\partial x^{2}},....\right)=0
\end{equation}

where

- x,y independent

- u(x,y) dependent

Fundamentally image a ODE as a means to track a single car, while
PDE track all the traffic in the city.

\subsubsection*{Types of ODEs}

ODEs are usually classified by 2 primary things. their order, aka
the degree of their derivative, and by whether they are linear or
non-linear. 

\paragraph*{First order ODEs}

are pretty self explanatory, they involve only the 1st derivative.
Here is a basic first order ODE:

\paragraph*{
\begin{equation}
\frac{dy}{dx}+y=x
\end{equation}
Second order ODEs}

involve UP to the 2nd derivative. Here is a example:

\paragraph*{
\begin{equation}
\frac{d^{2}y}{dx^{2}}+2\frac{dy}{dx}+y=0
\end{equation}
Higher order ODEs}

involve everything 3rd derivative or higher. It is unlikely to ever
appear in a 1st year analisis exam, but you never know. 

\paragraph*{Liniar and Non-liniear ODEs.}

An ODE is linear if the dependent variable and the derivatives are
in a linear form. Basically: they are not multiplied together. Anything
else is considered non-liniear. A linear ODE can be written in the
form:

\begin{equation}
a_{n}(x)\frac{d^{n}y}{dx^{n}}+a_{n-1}(x)\frac{d^{n-1}y}{dx^{n-1}}+...+a_{1}(x)\frac{dy}{dx}+a_{0}(x)y=f(x)
\end{equation}

- a(x) is a function of x

Here are some basic examples. We will go in much more detail when
solving ODEs.

$\frac{dy}{dx}+3y=x$, and $\frac{d^{2}y}{dx^{2}}+x\frac{dy}{dx}+y=sinx$

\[
Ineedtolearnbetterformating:)
\]
 A non-linear ODE is any ordinary differential equation that cannot
be written in the linear form shown earlier. This is because the dependent
variable or its derivatives are not linear. I will cover this more
later on in the chapter.

\subsubsection*{The weird classifications of ODEs}

\paragraph{A Homogeneous ODE,}

is a differential equation where L is a linear differential operator.

\paragraph{
\[
L[y]=0
\]
 A Non-Homogeneous ODE,}

is a differential equation where f(x) \ensuremath{\neq} 0

\paragraph{
\[
L[y]=f(x)
\]
 A Autonomous ODE,}

is a differential equation where the independent variable (usually
x or t) does not appear in the equation.

\paragraph*{
\[
\frac{d^{n}y}{dx^{n}}=F\left(y,y',...,y^{(n-1)}\right)
\]
A Non-Autonomous ODE,}

is a differential equation if the independent variable appears eq
(1).

{*}You can use multiple types at once, just use common sense to make
sure its right

\subsubsection*{Basic existence}

\subsection*{Dealing with ODEs}

\paragraph*{Separable ODE,}

can be written as 
\[
\frac{dy}{dx}=f(x)g(y)
\]
 Divide both sides by g(y), and multiply both sides by dx
\[
\frac{dy}{g(y)}=f(x)dx
\]
 Integrate both sides, make sure to keep the constant on the RHS
\[
\int\frac{dy}{g(y)}=\int f(x)dx
\]


\paragraph*{Integrating Method.}

Format the equation to fit the following before using the method
\[
\frac{dy}{dx}+P(x)y=Q(x)
\]
 Find the function $\mu(x)$ that will help simplify the problem.
(Symplify as much as possible here it will help a lot later on)
\[
\mu(x)=e^{\int P(x)dx}
\]
Multiply every term by the function $\mu(x)$ and using the product
rule calculate the derivative of the LHS
\[
\frac{d}{dx}(\mu(x)y)=\mu(x)Q(x)
\]
 Integrate both sides, remember the constant!
\[
y=\frac{1}{\mu(x)}\int\mu(x)Q(x)dx+C
\]

Quick note on integrating factor. A integrating factor is the method
used to solve derivative equation above. The \textgreek{μ}(x) also
called the integrating factor, works for any, first-order linear differential
equations. A function is derived by multiplying the equation with
\textgreek{μ}(x), which makes the left-hand side a derivative of \textgreek{μ}(x)y.

\paragraph*{Exact Method}

If a DE has a exact form (aka that there is a =0 and both terms are
split by a + ) the form:

\[
M(x,y)+N(x,y)\frac{dy}{dx}=0
\]
 OR it satisfies Clairaut’s Theorem, below: (both technically mean
the same thing) -btw Clairaut’s Theorem is most definitely not in
the syllabus 

\[
\frac{\partial M}{\partial y}=\frac{\partial N}{\partial x}
\]
Lets make a function that we will call $\Psi$ such that, $\varPsi_{x}=M(x,y)$
and $\Psi_{y}=N(x,y)$

Therefore we can write this now as

\[
\Psi_{x}+\Psi_{y}\frac{dy}{dx}=0
\]
we can start to find this function $\Psi$. So we will start to integrate
M with respect to x. (h(y) is a function of y)

\[
\Psi(x,y)=\int Mdx+h(y)
\]
 We can now differentiate $\Psi$ with respect to y

\[
\frac{\partial\Psi}{\partial y}=\frac{\partial}{\partial y}\left(\int Mdx\right)+h'(y)=N
\]
 Solve the integral for h(y)

\subsubsection*{Initial Value Problem}

Generally, IVPs are a DE and a \uline{initial condition} or condition's
which when used in unison they can be used to solve a function, that
will also fit the DE. The steps are pretty straight forward. 

1. Solve the DE

\[
y(x)=\int f(x)dx+C
\]

2. Use the initial condition, lets say that $y(x_{0})=y_{0}$

\[
y_{0}=\int f(x_{0})dx+C
\]

Where C is:

\[
C=y_{0}-V
\]
 {*}V is the value of the integral at $x_{0}$, therefore if we replace
C, the final answer is

\paragraph*{
\[
y(x)=\protect\int f(x)dx+(y_{0}-V)
\]
}

\subsection*{}

\paragraph{Interval of validity for Linear DE.}

The interval of validity for a Linear DE is largest around $x_{0}$where
$p(x)$ and $q(x)$ are continuous. Make sure to exclude discontinuities.

\paragraph{Interval of validity for Non-Linear DE,}

Solutions may be exponential to infinity or become undefined despite
$f(x,y)$ being smooth. Therefore check all the points where the solution
becomes undefined. Extend the interval on both sides of $x_{0}$ till
it is no longer possible.

\subsubsection*{Interpreting answers}

(this whole section is a bit useless, I may remove it if I don't find
any use for it son)

\paragraph*{Explicit Solution,}

is when the dependent variable is isolated and in terms of the independent
variable. e.g

\[
y=x^{2}+C
\]

(soluzione esplicita)

\paragraph*{Implicit Solution,}

is when the dependent variable is not explicitly isolated from the
independent. e.g

\[
x^{2}+y^{2}=C
\]

(soluzione implicita)

\paragraph{General Solution,}

is the solution containing all the possible solutions for the differential
equation, ie it keeps the constants, its the trivial form. (soluzione
generale)

\paragraph*{Particular Solution,}

is the solution which is a specific solution by locking the constants
by using the initial conditions, ie the C has a fixed value. (soluzione
particolare)

\paragraph*{Equilibrium Solution,}

is a solution which is constant because the dependent variable does
not change and therefore the derivative is zero. (soluzione di equilibrio)

\paragraph*{Parametric Solution,}

is a solution represented using a parameter like (t,u,z..) instead
of using x and y. eg.

\[
y(t)=\sqrt{t^{2}+C}
\]

(soluzione parametrica)

\subsubsection*{Second Order Equations-- needs a lot more work}

Homogeneous Equations with constant coefficients have the general
form:
\[
y''+ay'+by=0
\]
 Non-homogeneous Equations: Method of Undetermined Coefficients
\[
Finished
\]


\subsubsection*{Systems of Differential Equations}

\subsection*{Special theorems and Problems}

\subsubsection*{Cauchy's Problem}

Here we have a nth order ODE

\[
y^{(n)}(t)=f\left(t,y,y',....,y^{(n-1)}\right)
\]
 The Cauchy problem also known as the 
\end{document}
