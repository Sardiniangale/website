%% LyX 2.4.3 created this file.  For more info, see https://www.lyx.org/.
%% Do not edit unless you really know what you are doing.
\documentclass[american]{article}
\usepackage[LGR,T1]{fontenc}
\usepackage[utf8]{luainputenc}
\usepackage{esint}

\makeatletter

%%%%%%%%%%%%%%%%%%%%%%%%%%%%%% LyX specific LaTeX commands.
\DeclareRobustCommand{\greektext}{%
  \fontencoding{LGR}\selectfont\def\encodingdefault{LGR}}
\DeclareRobustCommand{\textgreek}[1]{\leavevmode{\greektext #1}}


\makeatother

\usepackage{babel}
\begin{document}
\title{Analisi 1}
\date{05/03/2025}
\author{Giacomo}
\maketitle

\part*{Pre-Derivatives}

\section*{Theorems, functions and axioms}

``Calvin: You know, I don’t think math is a science, I think it’s
a religion.

Hobbes: A religion?

Calvin: Yeah. All these equations are like miracles. You take two
numbers and when you add them, they magically become one NEW number!
No one can say how it happens. You either believe it or you don’t.
{[}Pointing at his math book{]} This whole book is full of things
that have to be accepted on faith! It’s a religion!''

\section*{Imaginary numbers}

\section*{Sums and Sequences}

\section*{Series}

\part*{Post-Derivatives}

\section*{Derivatives}

\textquotedbl It's dangerous to go alone! Take this.\textquotedbl{}

--- The Legend of Zelda

\section*{Integrals}

\textquotedbl I calculate the odds of this succeeding... never tell
me the odds!\textquotedbl{}

--- Han

\paragraph*{Reiman Integral}

\subsubsection*{Fundamental theorem of calculus}

\subsubsection*{Methods for integration}

Substitution. if $u=f(x)$, then $du=g'(x)dx$

\begin{equation}
\int f(g(x))g'(x)dx=\int f(u)du
\end{equation}
 Integration by parts

\begin{equation}
\int udv=uv-\int vdu
\end{equation}
 Lebiz rule for differentiation. If f(t) is continuous and g(x) is
differentiable
\[
f(x)=\int_{a}^{g(x)}f(t)dt
\]

If this is true, then:

\begin{equation}
F(x)'=f\left(g(x)\right)\cdot g'(x)
\end{equation}

If both the limits are dependent on x:

\[
I(x)=\int_{g_{1}(x)}^{g_{2}(x)}f(t)dt
\]

then:

\begin{equation}
I'(x)=f\left(g_{2}(x)\right)g_{2}'(x)-f\left(g_{1}(x)\right)g_{1}'(x)
\end{equation}
 Limit under a derviative 

\section*{Differential Equations}

\textquotedbl Would you tell me, please, which way I ought to go
from here?\textquotedbl{}

\textquotedbl That depends a good deal on where you want to get to,\textquotedbl{}
said the Cat.

\subsubsection*{ODEs and PDEs}

Ordinary Differential Equations (ODEs) are a differential equation
which has a single variable. ODEs have a general form:

\begin{equation}
F\left(x,y,\frac{dy}{dx},\frac{d^{2}y}{dx^{2}},...,\frac{d^{n}y}{dx^{n}}\right)=0
\end{equation}

where 

- x independent

- y dependent

Partial Differential Equations (PDEs) are a Differential equation
which has multiple independent variables. Instead of using the standard
d, they use partial derivatives (\ensuremath{\partial}) to show the
change with respect for multiple variables. The general form is:

\begin{equation}
F\left(x,y,u,\frac{\partial u}{\partial x},\frac{\partial u}{\partial y},\frac{\partial^{2}u}{\partial x^{2}},....\right)=0
\end{equation}

where

- x,y independent

- u(x,y) dependent

Fundamentally image a ODE as a means to track a single car, while
PDE track all the traffic in the city.

\subsubsection*{Types of ODEs}

ODEs are usually classified by 2 primary things. their order, aka
the degree of their derivative, and by whether they are linear or
non-linear. 

First order ODEs are pretty self explanatory, they involve only the
1st derivative. Here is a basic first order ODE:

\begin{equation}
\frac{dy}{dx}+y=x
\end{equation}

Second order ODEs involve UP to the 2nd derivative. Here is a example:

\begin{equation}
\frac{d^{2}y}{dx^{2}}+2\frac{dy}{dx}+y=0
\end{equation}

Higher order ODEs involve everything 3rd derivative or higher. It
is unlikely to ever appear in a 1st year analisis exam, but you never
know. 

Liniar and Non-liniear ODEs. An ODE is linear if the dependent variable
and the derivatives are in a linear form. Basically: they are not
multiplied together. Anything else is considered non-liniear. A linear
ODE can be written in the form: 

\begin{equation}
a_{n}(x)\frac{d^{n}y}{dx^{n}}+a_{n-1}(x)\frac{d^{n-1}y}{dx^{n-1}}+...+a_{1}(x)\frac{dy}{dx}+a_{0}(x)y=f(x)
\end{equation}

- a(x) is a function of x

Here are some basic examples. We will go in much more detail when
solving ODEs.

$\frac{dy}{dx}+3y=x$, and $\frac{d^{2}y}{dx^{2}}+x\frac{dy}{dx}+y=sinx$

(NON-LINIAR) To be added

\subsubsection*{The weird classifications of ODEs}

A Homogeneous ODE, is a differential equation where L is a linear
differential operator.

\[
L[y]=0
\]
 A Non-Homogeneous ODE, is a differential equation where f(x) \ensuremath{\neq}
0

\[
L[y]=f(x)
\]
 A Autonomous ODE, is a differential equation where the independent
variable (usually x or t) does not appear in the equation. 

\[
\frac{d^{n}y}{dx^{n}}=F\left(y,y',...,y^{(n-1)}\right)
\]
A Non-Autonomous ODE, is a differential equation if the independent
variable appears eq (1).

{*}You can be multiple classifications at once, use your head.

\subsubsection*{Dealing and solving ODEs }

A separable ODE can be written as 

\[
\frac{dy}{dx}=f(x)g(y)
\]

\[
\frac{dy}{g(y)}=f(x)dx
\]
 
\[
\int\frac{dy}{g(y)}=\int f(x)dx
\]
 Integrating Method

\[
\frac{dy}{dx}+P(x)y=Q(x)
\]
 
\[
\mu(x)=e^{\int P(x)dx}
\]

\[
\frac{d}{dx}(\mu(x)y)=\mu(x)Q(x)
\]

\[
y=\frac{1}{\mu(x)}\int\mu(x)Q(x)dx+C
\]
 Quick note on integrating factor.

A integrating factor is the method used to solve derivative equation
above. The \textgreek{μ}(x) also called the integrating factor, works
for any, first-order linear differential equations. A function is
derived by multiplying the equation with \textgreek{μ}(x), which makes
the left-hand side a derivative of \textgreek{μ}(x)y

\paragraph*{Second Order Equations}

Homogeneous Equations with Constant Coefficients
\[
N0t
\]
 Nonhomogeneous Equations: Method of Undetermined Coefficients
\[
Finished
\]


\subsection*{Special theorems and Problems}

\subsubsection*{Cauchy's Problem}
\end{document}
